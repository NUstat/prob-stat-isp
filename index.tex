% Options for packages loaded elsewhere
% Options for packages loaded elsewhere
\PassOptionsToPackage{unicode}{hyperref}
\PassOptionsToPackage{hyphens}{url}
\PassOptionsToPackage{dvipsnames,svgnames,x11names}{xcolor}
%
\documentclass[
  letterpaper,
  DIV=11,
  numbers=noendperiod]{scrreprt}
\usepackage{xcolor}
\usepackage{amsmath,amssymb}
\setcounter{secnumdepth}{5}
\usepackage{iftex}
\ifPDFTeX
  \usepackage[T1]{fontenc}
  \usepackage[utf8]{inputenc}
  \usepackage{textcomp} % provide euro and other symbols
\else % if luatex or xetex
  \usepackage{unicode-math} % this also loads fontspec
  \defaultfontfeatures{Scale=MatchLowercase}
  \defaultfontfeatures[\rmfamily]{Ligatures=TeX,Scale=1}
\fi
\usepackage{lmodern}
\ifPDFTeX\else
  % xetex/luatex font selection
\fi
% Use upquote if available, for straight quotes in verbatim environments
\IfFileExists{upquote.sty}{\usepackage{upquote}}{}
\IfFileExists{microtype.sty}{% use microtype if available
  \usepackage[]{microtype}
  \UseMicrotypeSet[protrusion]{basicmath} % disable protrusion for tt fonts
}{}
\makeatletter
\@ifundefined{KOMAClassName}{% if non-KOMA class
  \IfFileExists{parskip.sty}{%
    \usepackage{parskip}
  }{% else
    \setlength{\parindent}{0pt}
    \setlength{\parskip}{6pt plus 2pt minus 1pt}}
}{% if KOMA class
  \KOMAoptions{parskip=half}}
\makeatother
% Make \paragraph and \subparagraph free-standing
\makeatletter
\ifx\paragraph\undefined\else
  \let\oldparagraph\paragraph
  \renewcommand{\paragraph}{
    \@ifstar
      \xxxParagraphStar
      \xxxParagraphNoStar
  }
  \newcommand{\xxxParagraphStar}[1]{\oldparagraph*{#1}\mbox{}}
  \newcommand{\xxxParagraphNoStar}[1]{\oldparagraph{#1}\mbox{}}
\fi
\ifx\subparagraph\undefined\else
  \let\oldsubparagraph\subparagraph
  \renewcommand{\subparagraph}{
    \@ifstar
      \xxxSubParagraphStar
      \xxxSubParagraphNoStar
  }
  \newcommand{\xxxSubParagraphStar}[1]{\oldsubparagraph*{#1}\mbox{}}
  \newcommand{\xxxSubParagraphNoStar}[1]{\oldsubparagraph{#1}\mbox{}}
\fi
\makeatother

\usepackage{color}
\usepackage{fancyvrb}
\newcommand{\VerbBar}{|}
\newcommand{\VERB}{\Verb[commandchars=\\\{\}]}
\DefineVerbatimEnvironment{Highlighting}{Verbatim}{commandchars=\\\{\}}
% Add ',fontsize=\small' for more characters per line
\usepackage{framed}
\definecolor{shadecolor}{RGB}{241,243,245}
\newenvironment{Shaded}{\begin{snugshade}}{\end{snugshade}}
\newcommand{\AlertTok}[1]{\textcolor[rgb]{0.68,0.00,0.00}{#1}}
\newcommand{\AnnotationTok}[1]{\textcolor[rgb]{0.37,0.37,0.37}{#1}}
\newcommand{\AttributeTok}[1]{\textcolor[rgb]{0.40,0.45,0.13}{#1}}
\newcommand{\BaseNTok}[1]{\textcolor[rgb]{0.68,0.00,0.00}{#1}}
\newcommand{\BuiltInTok}[1]{\textcolor[rgb]{0.00,0.23,0.31}{#1}}
\newcommand{\CharTok}[1]{\textcolor[rgb]{0.13,0.47,0.30}{#1}}
\newcommand{\CommentTok}[1]{\textcolor[rgb]{0.37,0.37,0.37}{#1}}
\newcommand{\CommentVarTok}[1]{\textcolor[rgb]{0.37,0.37,0.37}{\textit{#1}}}
\newcommand{\ConstantTok}[1]{\textcolor[rgb]{0.56,0.35,0.01}{#1}}
\newcommand{\ControlFlowTok}[1]{\textcolor[rgb]{0.00,0.23,0.31}{\textbf{#1}}}
\newcommand{\DataTypeTok}[1]{\textcolor[rgb]{0.68,0.00,0.00}{#1}}
\newcommand{\DecValTok}[1]{\textcolor[rgb]{0.68,0.00,0.00}{#1}}
\newcommand{\DocumentationTok}[1]{\textcolor[rgb]{0.37,0.37,0.37}{\textit{#1}}}
\newcommand{\ErrorTok}[1]{\textcolor[rgb]{0.68,0.00,0.00}{#1}}
\newcommand{\ExtensionTok}[1]{\textcolor[rgb]{0.00,0.23,0.31}{#1}}
\newcommand{\FloatTok}[1]{\textcolor[rgb]{0.68,0.00,0.00}{#1}}
\newcommand{\FunctionTok}[1]{\textcolor[rgb]{0.28,0.35,0.67}{#1}}
\newcommand{\ImportTok}[1]{\textcolor[rgb]{0.00,0.46,0.62}{#1}}
\newcommand{\InformationTok}[1]{\textcolor[rgb]{0.37,0.37,0.37}{#1}}
\newcommand{\KeywordTok}[1]{\textcolor[rgb]{0.00,0.23,0.31}{\textbf{#1}}}
\newcommand{\NormalTok}[1]{\textcolor[rgb]{0.00,0.23,0.31}{#1}}
\newcommand{\OperatorTok}[1]{\textcolor[rgb]{0.37,0.37,0.37}{#1}}
\newcommand{\OtherTok}[1]{\textcolor[rgb]{0.00,0.23,0.31}{#1}}
\newcommand{\PreprocessorTok}[1]{\textcolor[rgb]{0.68,0.00,0.00}{#1}}
\newcommand{\RegionMarkerTok}[1]{\textcolor[rgb]{0.00,0.23,0.31}{#1}}
\newcommand{\SpecialCharTok}[1]{\textcolor[rgb]{0.37,0.37,0.37}{#1}}
\newcommand{\SpecialStringTok}[1]{\textcolor[rgb]{0.13,0.47,0.30}{#1}}
\newcommand{\StringTok}[1]{\textcolor[rgb]{0.13,0.47,0.30}{#1}}
\newcommand{\VariableTok}[1]{\textcolor[rgb]{0.07,0.07,0.07}{#1}}
\newcommand{\VerbatimStringTok}[1]{\textcolor[rgb]{0.13,0.47,0.30}{#1}}
\newcommand{\WarningTok}[1]{\textcolor[rgb]{0.37,0.37,0.37}{\textit{#1}}}

\usepackage{longtable,booktabs,array}
\usepackage{calc} % for calculating minipage widths
% Correct order of tables after \paragraph or \subparagraph
\usepackage{etoolbox}
\makeatletter
\patchcmd\longtable{\par}{\if@noskipsec\mbox{}\fi\par}{}{}
\makeatother
% Allow footnotes in longtable head/foot
\IfFileExists{footnotehyper.sty}{\usepackage{footnotehyper}}{\usepackage{footnote}}
\makesavenoteenv{longtable}
\usepackage{graphicx}
\makeatletter
\newsavebox\pandoc@box
\newcommand*\pandocbounded[1]{% scales image to fit in text height/width
  \sbox\pandoc@box{#1}%
  \Gscale@div\@tempa{\textheight}{\dimexpr\ht\pandoc@box+\dp\pandoc@box\relax}%
  \Gscale@div\@tempb{\linewidth}{\wd\pandoc@box}%
  \ifdim\@tempb\p@<\@tempa\p@\let\@tempa\@tempb\fi% select the smaller of both
  \ifdim\@tempa\p@<\p@\scalebox{\@tempa}{\usebox\pandoc@box}%
  \else\usebox{\pandoc@box}%
  \fi%
}
% Set default figure placement to htbp
\def\fps@figure{htbp}
\makeatother





\setlength{\emergencystretch}{3em} % prevent overfull lines

\providecommand{\tightlist}{%
  \setlength{\itemsep}{0pt}\setlength{\parskip}{0pt}}



 


\KOMAoption{captions}{tableheading}
\makeatletter
\@ifpackageloaded{bookmark}{}{\usepackage{bookmark}}
\makeatother
\makeatletter
\@ifpackageloaded{caption}{}{\usepackage{caption}}
\AtBeginDocument{%
\ifdefined\contentsname
  \renewcommand*\contentsname{Table of contents}
\else
  \newcommand\contentsname{Table of contents}
\fi
\ifdefined\listfigurename
  \renewcommand*\listfigurename{List of Figures}
\else
  \newcommand\listfigurename{List of Figures}
\fi
\ifdefined\listtablename
  \renewcommand*\listtablename{List of Tables}
\else
  \newcommand\listtablename{List of Tables}
\fi
\ifdefined\figurename
  \renewcommand*\figurename{Figure}
\else
  \newcommand\figurename{Figure}
\fi
\ifdefined\tablename
  \renewcommand*\tablename{Table}
\else
  \newcommand\tablename{Table}
\fi
}
\@ifpackageloaded{float}{}{\usepackage{float}}
\floatstyle{ruled}
\@ifundefined{c@chapter}{\newfloat{codelisting}{h}{lop}}{\newfloat{codelisting}{h}{lop}[chapter]}
\floatname{codelisting}{Listing}
\newcommand*\listoflistings{\listof{codelisting}{List of Listings}}
\usepackage{amsthm}
\theoremstyle{remark}
\AtBeginDocument{\renewcommand*{\proofname}{Proof}}
\newtheorem*{remark}{Remark}
\newtheorem*{solution}{Solution}
\newtheorem{refremark}{Remark}[chapter]
\newtheorem{refsolution}{Solution}[chapter]
\makeatother
\makeatletter
\makeatother
\makeatletter
\@ifpackageloaded{caption}{}{\usepackage{caption}}
\@ifpackageloaded{subcaption}{}{\usepackage{subcaption}}
\makeatother
\usepackage{bookmark}
\IfFileExists{xurl.sty}{\usepackage{xurl}}{} % add URL line breaks if available
\urlstyle{same}
\hypersetup{
  pdftitle={Probability and Statistics for ISP},
  pdfauthor={T. A. Severini},
  colorlinks=true,
  linkcolor={blue},
  filecolor={Maroon},
  citecolor={Blue},
  urlcolor={Blue},
  pdfcreator={LaTeX via pandoc}}


\title{Probability and Statistics for ISP}
\usepackage{etoolbox}
\makeatletter
\providecommand{\subtitle}[1]{% add subtitle to \maketitle
  \apptocmd{\@title}{\par {\large #1 \par}}{}{}
}
\makeatother
\subtitle{Course Notes for Statistics 383}
\author{T. A. Severini}
\date{2025-12-31}
\begin{document}
\maketitle

\renewcommand*\contentsname{Table of contents}
{
\hypersetup{linkcolor=}
\setcounter{tocdepth}{2}
\tableofcontents
}

\bookmarksetup{startatroot}

\chapter*{Preface}\label{preface}
\addcontentsline{toc}{chapter}{Preface}

\markboth{Preface}{Preface}

This is a Quarto book.

To learn more about Quarto books visit
\url{https://quarto.org/docs/books}.

\begin{Shaded}
\begin{Highlighting}[]
\DecValTok{1} \SpecialCharTok{+} \DecValTok{1}
\end{Highlighting}
\end{Shaded}

\begin{verbatim}
[1] 2
\end{verbatim}

\bookmarksetup{startatroot}

\chapter{Probability}\label{sec-chap1}

\newcommand{\cip}{\xrightarrow[]{p}}
\renewcommand{\P}{\mbox{P}}
\newcommand{\E}{\mbox{E}}
\newcommand{\Fhat}{\hat{F}}
\newcommand{\intii}{\int_{-\infty}^\infty}
\newcommand{\Ybar}{\bar{Y}}
\newcommand{\Var}{\mbox{Var}}
\newcommand{\lam}{\lambda}
\newcommand{\Ph}{\widehat{\mbox{P}}}
\newcommand{\I}{\mbox{I}}
\newcommand{\cid}{\xrightarrow[]{D}}

\newcommand{\bfmu}{\bm{\mu}}  
\newcommand{\bfw}{\bm{w}}
\newcommand{\bfR}{\bm{R}}
\newcommand{\bfone}{\bm{1}}
\newcommand{\bfx}{\bm{x}}
\newcommand{\bfy}{\bm{y}}
\newcommand{\bfz}{\bm{z}}
\newcommand{\bfV}{\bm{\mathcal{V}}}
\newcommand{\bfVm}{\bm{V}}
\newcommand{\bfq}{\bm{q}}
\newcommand{\bfC}{\bm{C}}
\newcommand{\bfI}{\bm{I}}
\newcommand{\bfdelta}{\bm{\delta}}
\newcommand{\bfbeta}{\bm{\beta}}
\newcommand{\bfS}{\bm{S}}
\newcommand{\bfalpha}{\bm{\alpha}}
\newcommand{\bfepsilon}{\bm{\epsilonilon}}
\newcommand{\bfd}{\bm{d}}
\newcommand{\bfSigma}{\bm{\Sigma}}
\newcommand{\tbfSigma}{\tilde{\bfSigma}}
\newcommand{\bfzero}{\bm{0}}
\newcommand{\bfA}{\bm{A}}
\newcommand{\bfa}{\bm{a}}
\newcommand{\bfB}{\bm{B}}
\newcommand{\bfh}{\bm{h}}
\newcommand{\bfv}{\bm{v}}
\newcommand{\bfX}{\bm{X}}
\newcommand{\bfu}{\bm{u}}
\newcommand{\bfg}{\bm{g}}
\newcommand{\bfb}{\bm{b}}
\newcommand{\muhat}{\hat{\mu}}
\newcommand{\bfM}{\bm{M}}
\newcommand{\bfGamma}{\bm{\Gamma}}

\newcommand{\p}{{q}}
\newcommand{\bfF}{\bm{F}}
\newcommand{\phat}{\hat{p}}
\newcommand{\that}{\hat{\theta}}
\newcommand{\bbar}{\bar{b}}
\newcommand{\mhat}{\hat{m}}
\newcommand{\knumer}{\frac{1}{nh} \sum_{j=1}^n K(\frac{x - X_j}{h})Y_j }
\newcommand{\given}{{\ | \ }}
\newcommand{\fhat}{\hat{f}}
\newcommand{\Cov}{\mbox{Cov}}
\newcommand{\dmX}{\mathbb{{X}}}
\newcommand{\covS}{\mathbb{S}}
\newcommand{\bfe}{\bm{e}}
\newcommand{\bfell}{\bm{\ell}}
\newcommand{\bfY}{\bm{Y}}
\newcommand{\bfL}{\bm{L}}
\newcommand{\bfPsi}{\bm{\Psi}}
\newcommand{\bfQ}{\bm{Q}}
\newcommand{\bfT}{\bm{T}}
\newcommand{\X}{\mathcal{X}}
\newcommand{\Y}{\mathcal{Y}}
\newcommand{\xbar}{\bar{x}}

\newcommand{\theto}{\theta_0}

\newcommand{\bfZ}{\bm{Z}}
\newcommand{\bfP}{\bm{P}}
\newcommand{\bfD}{\bm{D}}
\newcommand{\bfO}{\bm{O}}
\newcommand{\bfLambda}{\bm{\Lambda}}
\newcommand{\bfc}{\bm{c}}
\newcommand{\bfE}{\bm{E}}
\newcommand{\bfU}{\bm{U}}
\newcommand{\bff}{\bm{f}}
\newcommand{\bfH}{\bm{H}}
\newcommand{\Xbar}{\bar{X}}

\newcommand{\ybar}{\bar{y}}

\newcommand{\calP}{\mathcal{P}}
\newcommand{\calY}{\mathcal{Y}}
\newcommand{\calX}{\mathcal{X}}
\newcommand{\calR}{\mathcal{R}}
\newcommand\BB{\rule[-1ex]{0pt}{0pt}}  
\newcommand{\psihat}{{\widehat{\psi}}}
\newcommand{\thetast}{\theta^*}
\newcommand{\thetst}{\theta^*}

\newcommand\T{\rule{0pt}{2.6ex}}       
\newcommand\B{\rule[-1ex]{0pt}{0pt}}

\section{Experiments}\label{experiments}

The starting point for probability theory is the concept of an
\emph{experiment}. The term experiment may actually refer to a physical
experiment in the usual sense, but more generally we will refer to
something as an experiment when it has the following properties:

\begin{itemize}
\item
  There is a well-defined set of possible outcomes of the experiment;
\item
  Each time the experiment is performed exactly one of the possible
  outcomes occurs;
\item
  The outcome that occurs is governed by some chance mechanism.
\end{itemize}

Let \(\Omega\) denote the \emph{sample space} of the experiment, the set
of possible outcomes of the experiment; the term \emph{outcome space} is
also used. We will refer to the elements of \(\Omega\) as \emph{basic
outcomes} and use the symbol \(\omega\) to denote a generic basic
outcome.

\phantomsection\label{onedie1}{} Consider the experiment in which we
roll a die. Then \[\Omega  = \left\{1, 2, 3, 4, 5, 6 \right\}\] where,
for example, \(1\) denotes the outcome that we roll a \(1\).

Consider the experiment in which we choose a number from the interval
\((0, 1)\). Then \(\Omega = (0, 1)\).

\phantomsection\label{urn3}{} Suppose that we have an urn that contains
three balls, two red balls and one black ball. Consider the experiment
in which we successively choose two balls from the urn, that is, the
balls are chosen in such a way that we know which ball was chosen first.

Then the sample space for the experiment can be written
\[\Omega = \left\{ (R, R), \ (R, B), \ (B, R) \right\},\] where, for
example, \((R, B)\) means that the first ball selected is red and the
second ball selected is black.

Now suppose that the order in which the balls were selected is not
recorded. Then the sample space of the experiment is given by
\[\Omega = \left\{ \{R, R\}, \ \{R, B\} \right\},\] where, for example,
\(\{R, B\}\) means that one red ball is selected and one black ball is
selected.

\section{Events}\label{events}

Consider an experiment with sample space \(\Omega\). A subset \(A\) of
\(\Omega\) is called an \emph{event}. Let \(A\) be an event. Then, for
each \(\omega\in\Omega\), either \(\omega\in A\) or \(\omega\notin A\).
That is, when the experiment is performed, either \(A\) occurs (the
observed outcome is in \(A\)) or it doesn't occur (the observed outcome
is not in \(A\)).

\phantomsection\label{onedie2}{} Consider the experiment in which we
roll a die. Then \[A = \left\{2, 4, 6 \right\}\] is the event that we
roll an even number.

The event that we roll a number less than or equal to \(3\) is given by
\[B = \left\{1, 2, 3 \right\}.\]

The event that we roll a \(5\) is given by \[C = \left\{ 5 \right\}.\]

The event that we do not roll an even number, that is, that we roll an
odd number is given by \(A^c\), the complement of \(A\). Thus,
\[A^c = \left\{1, 3, 5 \right\}.\]

Consider the experiment in which two balls are drawn successively from
an urn containing two red balls and one black ball; the sample space for
this experiment is given in Example \hyperref[urn3]{{[}urn3{]}}.

Let \(A\) denote the event that exactly one red ball is selected. Then
\[A  = \left\{ (R, B), \ (B, R) \right\}.\]

Because events are defined in terms of sets, sets play a central role in
probability theory. Here are few basic properties. Let \(A, B, C\) be
subsets of a sample space \(\Omega\); that is, let \(A, B, C\) be
events. Recall that \(A \cup B\), the \emph{union} of \(A\) and \(B\),
is the set consisting of all elements that are either in \(A\), in
\(B\), or in both \(A\) and \(B\); \(A\cap B\), the \emph{intersection}
of \(A\) and \(B\), is the set consisting of all elements that are in
both \(A\) and \(B\). Then

\begin{itemize}
\item
  \((A \cup B)\cap C = (A \cap C) \cup (B \cap C)\)
\item
  \((A \cap B) \cup C = (A \cup C) \cap (B \cup C)\)
\item
  \((A\cup B)^c = A^c \cap B^c\)
\item
  \((A \cap B)^c = A^c \cup B^c\).
\end{itemize}

If these properties are unfamiliar, you can show why they hold using
Venn diagrams. E.g., consider \((A \cup B)^c = A^c \cap B^c\). Figures
\hyperref[DeMorg1]{1.1} -- \hyperref[DeMorg3]{1.3} contain Venn diagrams
of \((A\cup B)^c\), \(A^c\), and \(B^c\), respectively. From these
diagrams, we can see that \((A \cup B)^c = A^c \cap B^c\).

\begin{figure}

{\centering \includegraphics[width=0.5\linewidth,height=\textheight,keepaspectratio]{demorgan1.pdf}

}

\caption{\((A \cup B)^c\)}

\end{figure}%

\begin{figure}

{\centering \includegraphics[width=0.5\linewidth,height=\textheight,keepaspectratio]{demorgan2new.pdf}

}

\caption{\(A^c\)}

\end{figure}%

\begin{figure}

{\centering \includegraphics[width=0.5\linewidth,height=\textheight,keepaspectratio]{demorgan3.pdf}

}

\caption{\(B^c\)}

\end{figure}%

\section{Probability Functions}\label{prob_fcns}

Consider an experiment with sample space \(\Omega\); recall that the
outcome of an experiment depends on some ``chance mechanism". It follows
that whether or not an event \(A\) occurs depends on that chance
mechanism and we use probability theory to describe the likelihood that
a given event occurs.

Therefore, associated with each event \(A\) is a probability
\(\mbox{P}(A)\). Here \(\mbox{P}\) is a function defined on subsets of
\(\Omega\) and taking values in the interval \([0, 1]\). The function
\(\mbox{P}\) is required to have certain properties:

\begin{itemize}
\item
  \(\mbox{P}(\Omega) = 1\)
\item
  If \(A\) and \(B\) are disjoint subsets of \(\Omega\), that is,
  \(A\cap B = \emptyset\), then
  \(\mbox{P}(A \cup B) = \mbox{P}(A) + \mbox{P}(B)\).
\item
  If \(A_1, A_2, \ldots,\) are disjoint subsets of \(\Omega\), then
  \[\mbox{P}( \cup_{n=1}^\infty A_n) = \sum_{n=1}^\infty \mbox{P}(A_n).\]
\end{itemize}

Note that when subsets of \(\Omega\) are disjoint, the corresponding
events are said to be \emph{mutually exclusive}.

\phantomsection\label{uni_ex}{} Suppose that \(\Omega = (0, 1)\) and
suppose that the probability of any interval in \(\Omega\) is the length
of the interval. More generally, we may take the probability of a subset
\(A\) of \(\Omega\) to be \[\mbox{P}(A) = \int_{A} dx.\]

For example, \[\mbox{P}\left( (0.2, 0.7) \right) = 0.5.\]

Consider the experiment of rolling one die, as discussed in Examples
\hyperref[onedie1]{{[}onedie1{]}} and \hyperref[onedie2]{{[}onedie2{]}}
and let \(\Omega\) denote the sample space of the experiment.

For \(A \subset \Omega\), let \(\mbox{P}(A) = |A|/6\), the number of
elements in \(A\), divided by \(6\).

E.g., the probability of rolling an even number is \(1/2\) and the
probability of rolling a number greater than or equal to \(5\) is
\(1/3\).

Note that, when an event consists of a single basic outcome \(\omega\),
we will write the probability of the event as \(\mbox{P}(\omega)\),
rather than as \(\mbox{P}(\left\{ \omega \right\})\), which is
technically correct (because the argument of the probability function
should be a set).

For instance, in the previous example, the probability of rolling a
\(6\) will be written as \(\mbox{P}(6)\) instead of as
\(\mbox{P}(\{ 6 \})\).

When \(\Omega\) is a countable set, then, by properties (P2) and (P3),
the probability of any event is given by the sum of the probabilities of
the basic outcomes corresponding to the event:
\[\mbox{P}(A) = \sum_{\omega \in A} \mbox{P}( \omega).\]

\phantomsection\label{binom_ex}{} Consider an experiment with sample
space
\[\Omega = \left\{(0, 0), \ (1, 0), \  (0, 1), \ (1, 1) \right\}.\]

For \(\omega = (x_1, x_2) \in \Omega\), take \[\begin{aligned}
\mbox{P}(\omega) &= \theta^{x_1} (1-\theta)^{1-x_1} \theta^{x_2} (1-\theta)^{1-x_2} \\
&= \theta^{x_1 + x_2} (1-\theta)^{2 - x_1 - x_2}
\end{aligned}\] where \(0 < \theta < 1\) is a given constant.

Thus, the four elements of \(\Omega\) have probabilities
\((1-\theta)^2, \theta(1-\theta), \theta(1-\theta), \theta^2\),
respectively.

Let \(A\) denote the event that exactly \(1\) one is observed; then
\[A = \left\{ (0, 1), \ (1, 0) \right\}.\] It follows that the
probability of \(A\) is the sum of the probabilities of the two basic
outcomes in \(A\):
\[\mbox{P}(A) = \mbox{P}\left( (1, 0) \cup (0, 1) \right) = \mbox{P}\left((1, 0) \right) + \mbox{P}\left( (0, 1) \right) = \theta(1-\theta) + \theta(1-\theta) = 2\theta(1-\theta).\]

\subsection*{Some implications of (P1) --
(P3)}\label{some-implications-of-p1-p3}
\addcontentsline{toc}{subsection}{Some implications of (P1) -- (P3)}

There are a number of straightforward consequences of properties
(P1)-(P3). For instance, because \(\Omega \cup \emptyset = \Omega\) and
\(\Omega \cap \emptyset = \emptyset\), by (P2)
\[\mbox{P}(\Omega) = \mbox{P}(\Omega) + \mbox{P}(\emptyset);\] it
follows that \(\mbox{P}(\emptyset) = 0\).

Let \(A^c\) denote the complement of an set \(A\subset \Omega\). Then
\(A \cup A^c = \Omega\) and \(A \cap A^c = \emptyset\). It follows from
(P2) that \[\mbox{P}(\Omega) = \mbox{P}(A) + \mbox{P}(A^c);\] it now
follows from (P1) that \[\mbox{P}(A^c) = 1 - \mbox{P}(A).\]

Suppose that \(A_1\) and \(A_2\) are subsets of \(\Omega\) that are not
necessarily disjoint. Then
\[\mbox{P}(A_1 \cup A_2) =\mbox{P}(A_1) + \mbox{P}(A_2) - \mbox{P}(A_1 \cap 
A_2).\]

This important result is a little more difficult to prove than the
others we have considered. First note that \(A_1 \cup A_2\) can be
written as the union of three sets, \(A_1 \cap A_2\),
\(A_1 \cap A_2^c\), and \(A_1^c \cap A_2\). Furthermore, these three
sets are disjoint. An example of this fact is given in the Venn diagram
in Figure \hyperref[venn]{1.4}. In that diagram, the blue region is
\(A_1 \cap A_2^c\), the yellow region is \(A_1^c \cap A_2\), and the
brown region is \(A_1 \cap A_2\); combining these three regions forms
\(A_1 \cup A_2\).

\begin{figure}

{\centering \includegraphics[width=0.75\linewidth,height=\textheight,keepaspectratio]{vplot.pdf}

}

\caption{Venn Diagram Used to Illustrate
\(A_1 \cup A_2 = (A_1 \cap A_2) \cup (A_1 \cap A_2^c) \cup
(A_1^c \cap A_2)\)}

\end{figure}%

From the Venn diagram we can also see that
\[A_1 = (A_1 \cap A_2^c) \cup (A_1 \cap A_2) \ \ \text{ and } \ \ 
A_2 = (A_1^c \cap A_2) \cup (A_1 \cap A_2).\]

It follows from two applications of (P2) that \[\label{probeq1}
 \mbox{P}(A_1 \cup A_2) = \mbox{P}(A_1 \cap A_2) + \mbox{P}(A_1 \cap A_2^c) + \mbox{P}(A_1^c \cap A_2)\]
and also that
\[\mbox{P}(A_1) = \mbox{P}(A_1 \cap A_2^c) + \mbox{P}(A_1 \cap A_2) \ \ \text{ and } \ \ 
\mbox{P}(A_2) = \mbox{P}(A_1^c \cap A_2) + \mbox{P}(A_1 \cap A_2).\]

From the last two equations, we see that
\[\mbox{P}(A_1 \cap A_2^c) = \mbox{P}(A_1) - \mbox{P}(A_1 \cap A_2) \ \ \text{ and } \ \ 
\mbox{P}(A_1^c \cap A_2) = \mbox{P}(A_2) - \mbox{P}(A_1 \cap A_2).\]
Substituting these expressions into the right-hand side of equation
\hyperref[probeq1]{{[}probeq1{]}}, it follows that \[\begin{aligned}
\mbox{P}(A_1 \cup A_2) &= \mbox{P}(A_1 \cap A_2) + \mbox{P}(A_1) - \mbox{P}(A_1 \cap A_2) + \mbox{P}(A_2) - \mbox{P}(A_1 \cap A_2) \\
&= \mbox{P}(A_1) + \mbox{P}(A_2) - \mbox{P}(A_1 \cap A_2). 
\end{aligned}\]

Consider an experiment with sample space \(\Omega\) and events
\(A_1, A_2\). Let \(A_1 \setminus A_2\) denote the elements of \(A_1\)
that are not in \(A_2\).

Suppose that \(A_2\subset A_1\). Then
\[\mbox{P}(A_1 \setminus A_2) = \mbox{P}(A_1) - \mbox{P}(A_2).\]

In general, \[A_1 = (A_1 \cap A_2) \cup (A_1 \cap A_2^c) = 
(A_1 \cap A_2) \cup (A_1 \setminus A_2).\] Note that
\[A_1\cap A_2 \ \ \text{ and } \ \ A_1 \cap A_2^c\] are disjoint and
that \[A_1 \cap A_2^c = A_1 \setminus A_2.\] Hence,
\[\mbox{P}(A_1) = \mbox{P}(A_1 \cap A_2) + \mbox{P}(A_1 \setminus A_2)\]
so that
\[\mbox{P}(A_1 \setminus A_2) = \mbox{P}(A_1) - \mbox{P}(A_1 \cap A_2).\]

\subsection*{Interpretation of
probability}\label{interpretation-of-probability}
\addcontentsline{toc}{subsection}{Interpretation of probability}

Although we have described the properties of a probability function,
nothing has been said about what the probability function is measuring.
In a mathematical sense, that is irrelevant -- a probability function is
defined by its properties and any function satisfying those properties
can be used to calculate a ``probability".

However, in order to better understand the mathematical results, and to
develop some intuition regarding probability theory, it is useful to
have some notion of what is meant by ``probability". Several different
interpretations of probability are used in applications. The most
common, and the one we will use here, is the interpretation of
probability as a''limiting relative frequency".

Consider an experiment with sample space \(\Omega\) and let \(A\) denote
an event. According to the limiting relative frequency interpretation of
probability, the statement that \(A\) has probability \(0.4\) (for
example), means that if the experiment is repeated a large number of
times then in about \(40\%\) of those experiments the event \(A\) will
occur.

More formally, let \(N_n(A)\) denote the number of times the event \(A\)
occurs in \(n\) repetitions of the experiment. Then
\[\mbox{P}(A) = \lim_{n\to\infty} \frac{N_n(A)}{n};\] the right-hand
side of this expression is often described as a ``limiting relative
frequency".

\section{Sampling from a Finite
Population}\label{sampling-from-a-finite-population}

A particularly simple, but useful, case occurs when the sample space of
the experiment, \(\Omega\), is a finite set and each \(\omega\in\Omega\)
has the same probability.

Write
\[\Omega = \left\{ \omega_1, \omega_2, \ldots, \omega_m \right\};\] note
that any subset of \(\Omega\) can be written as the union of sets of the
form \(\{ \omega_j \}\), which are disjoint. Let
\(c = \mbox{P}(\omega_j)\) denote the common value of the probability of
each element of \(\Omega\). Then, because \(\mbox{P}(\Omega) = 1\) and
\[\mbox{P}(\Omega) = \mbox{P}(\omega_1) + \mbox{P}(\omega_2) + \cdots +  \mbox{P}(\omega_m) = m c\]
we must have \(c = 1/|\Omega|\) where \(|\Omega|\) denotes the
cardinality of \(\Omega\), that is, the number of elements in
\(\Omega\).

Furthermore, for any \(A \subset \Omega\),
\[\mbox{P}(A) = \sum_{\omega\in A} \mbox{P}(\omega) =  {|A| \over |\Omega|}.\]
Thus, the problem of determining \(\mbox{P}(A)\) is essentially the
problem of counting the number of elements in the set \(A\) and the
number of elements in \(\Omega\). The subfield of mathematics concerned
with counting the number of elements in a set is known as
\emph{combinatorics}.

In some cases, such as the one in the following example, the counting
needed is relatively straightforward.

\phantomsection\label{dice1}{}

Consider the experiment of rolling \(2\) dice, one at a time. The sample
space of the experiment can be written
\[\Omega =  \left\{ (1, 1), (1, 2), \ldots, (1, 6), (2, 1), \ldots, (2, 6), \ldots, (6, 1), \ldots, (6, 6) \right\}\]
so that it has \(36\) elements. Suppose the dice are ``fair", in the
sense that each element of \(\Omega\) is equally likely.

Let \(A\) denote the event that result of the dice rolling is
``doubles", i.e., the two numbers rolled are equal, and suppose that we
are interested in the probability of \(A\).

As noted previously, the sample space \(\Omega\) has \(36\) elements.
The event \(A\) has \(6\) elements, \((1, 1), (2, 2), \ldots, (6, 6)\).
Thus, the probability of rolling doubles is \(6/36 = 1/6\).

In other cases, the counting needed is more complicated and it is useful
to apply one of the many well-known results that are used to solve such
counting problems. Here we consider only a few simple ones.

\subsection*{Counting principle}\label{counting-principle}
\addcontentsline{toc}{subsection}{Counting principle}

Many results in combinatorics are based on the \emph{counting
principle}. Let \(A\) and \(B\) denote finite sets and let \(A\times B\)
denote the Cartesian product of \(A\) and \(B\), that is, the set of the
form \[A\times B = \left\{ (a, b): \ a\in A, \ b\in B \right\}.\] Then
\[|A\times B |= |A| \ |B|.\]

Thus, if a task can be completed in \(r\) stages and there are \(n_j\)
ways to complete the \(j\)th stage, \(j=1, 2, \ldots, r\), then there
are \[n_1\times n_2 \times \cdots \times n_r\] ways to complete the
task.

\subsection*{Permutations and
combinations}\label{permutations-and-combinations}
\addcontentsline{toc}{subsection}{Permutations and combinations}

A \emph{permutation} of \(n\) distinct objects is an ordering of them.

The possible permutations of \(a, b, c\) are
\[abc, \ bac, \ cab, \ acb, \ bca, \ cba.\]

The number of permutations of \(n\) distinct objects can be found using
the counting principle, breaking the problem into stages. Note that
there are \(n\) ways to choose the first object, \(n-1\) ways to choose
the second object, and so on, until there is only \(1\) way to choose
the last object. Hence, there are
\[n\cdot (n-1) \cdots (2)\cdot (1) = n!\] ways to order the \(n\)
objects. That is, there are \(n!\) possible permutations of \(n\)
distinct objects.

In some cases, we might be interested in ordered \emph{samples} from a
given set.

The possible ordered samples of size \(2\) from the set \(\{a, b, c \}\)
are \[(a, b),\ (a, c),\  (b, a),\  (b, c), \ (c, a), \ (c, b).\]

The number of possible ordered samples of size \(k\) from a set of \(n\)
distinct elements can be found using the counting principle. There are
\(n\) ways to choose the first element, \(n-1\) ways to choose the
second element, and so on. However, in contrast to permutations, here we
stop selecting elements after the \(k\)th selection. Therefore, there
\[n (n-1) \cdots (n-k+1)\] ordered samples of size \(k\) from a set of
\(n\) elements; note there are \(k\) terms in this product. The
expression \(n(n-1)\cdots (n-k+1)\) is often denoted by \((n)_k\); it
can also be written as \[(n)_k = \frac{n!}{(n-k)!}.\]

\phantomsection\label{urn0}{} Consider an urn containing \(5\) balls,
\(2\) of which are black and \(3\) of which are red. Suppose that \(2\)
balls are randomly selected from the urn, without replacement; that is,
after the first ball is selected, it is not returned to the urn for the
second selection. Thus, there are \((5)_2 = 5(4) = 20\) basic outcomes
in \(\Omega\); ``randomly selected" means that any ordered pair of \(2\)
balls is equally likely to be selected.

Let \(A\) denote the event that a red ball is selected followed by a
black ball. Find \(\mbox{P}(A)\). Because each basic outcome is assumed
to have the same probability, \[\mbox{P}(A) = \frac{|A|}{|\Omega|}.\]
Thus, we need to count the number of basic outcomes in of \(A\).

To find the number of basic outcomes in \(A\), we use the facts that
there are \(3\) ways to choose the first (red) ball and \(2\) ways to
choose the second (black) ball. Thus, there are \((3)_2=3(2) = 6\) basic
outcomes in \(A\). It follows that
\[\mbox{P}(A) = \frac{6}{20} = \frac{3}{10}.\]

Now suppose that we are interested in the possible \emph{combinations}
of \(k\) objects chosen from a set of \(n\) distinct objects. When
considering combinations, the order of the objects is irrelevant; we are
interested only in the set of \(k\) objects.

The possible combinations of \(2\) elements chosen from the set
\(\{a, b, c \}\) are given by \[\{a, b \}, \ \{a, c \}, \ \{b, c \}.\]

The number of possible combinations of \(k\) objects chosen from a set
of \(n\) distinct objects is denoted by \[{n \choose k},\] read as
``\(n\) choose \(k\)".

To find an expression for \(n\) choose \(k\), consider choosing an
\textbf{ordered} sample of size \(k\) from \(n\) distinct elements.

This can be done in two steps:

\begin{enumerate}
\def\labelenumi{\arabic{enumi}.}
\item
  choose \(k\) elements from \(n\)
\item
  order the \(k\) elements
\end{enumerate}

We know that

\begin{itemize}
\item
  there are \((n)_k\) ordered samples of size \(k\) from \(n\) distinct
  elements
\item
  there are \(n\) choose \(k\) ways to choose \(k\) elements from \(n\)
\item
  there are \(k!\) ways to order \(k\) elements
\end{itemize}

Therefore, we must have \[(n)_k = {n \choose k}\, k!\] so that
\[{n \choose k} =  \frac{(n)_k}{k!} = \frac{n!}{k!\, (n-k)!}.\]

Terms of the form \({n \choose k}\) are often called the \emph{binomial
coefficients}, because of the binomial formula:
\[(x+y)^n = \sum_{j=0}^n {n \choose j} x^j y^{n-j}\] for all real
numbers \(x, y\) and all positive integers \(n\).

Consider the framework of Example \hyperref[urn0]{{[}urn0{]}}: there is
an urn with \(2\) black balls and \(3\) red balls and \(2\) balls are
randomly selected from the urn. Let \(B\) denote the event that \(2\)
red balls are selected. Find \(\mbox{P}(B)\).

The basic outcomes here are the sets of two balls selected from the urn.
Because each basic outcome is assumed to have the same probability,
\[\mbox{P}(B) = \frac{|B|}{|\Omega|}.\]

The number of elements in \(\Omega\) is the number of ways to select
\(2\) balls from the set of \(5\), given by \[{5 \choose 2} = 10.\] The
number of elements in \(B\) is the number of ways to choose \(2\) red
balls from the set of \(3\) red balls, given by \[{3 \choose 2} = 3.\]
Thus, \(\mbox{P}(B) = 3/10\).

Suppose that \(5\) cards are dealt from a well-shuffled deck of playing
cards. Recall that, in such a deck, there are \(52\) cards and each card
falls into one of four suits (\(13\) cards in each suit).

What is the probability that all \(5\) cards are of the same suit (i.e.,
a ``flush", in poker)?

There are \[{52 \choose 5}\] ways to choose \(5\) cards from a deck of
\(52\). To find the number of ways in which \(5\) cards can be chosen
from one suit, we can use the counting principle: there are \(4\) ways
to choose the suit and, given the suit, there are \[{13 \choose 5}\]
ways to choose the \(5\) cards from the suit. Therefore, there are
\[4 {13 \choose 5}\] ways to choose \(5\) cards from one suit.

It follows that the probability of being dealt \(5\) cards from one suit
is \[\begin{aligned}
\frac{4 {13 \choose 5}}{{52 \choose 5}} &= \frac{4 \frac{13!}{8!\, 5!}}{\frac{52!}{47!\, 5!} }\\
&= 4 \frac{(13)(12)(11)(10)(9)}{(52)(51)(50)(49)(48)}  \\
&= 0.00198. 
\end{aligned}\]

In some cases, it is easier to find the probability of an event \(A\) by
finding the probability of \(A^c\) and using the fact that
\(\mbox{P}(A) = 1 - \mbox{P}(A^c)\). In fact, this simple result often
converts a complicated problem into a relatively easy one.

Suppose that \(n\) cards are dealt from a well-shuffled deck of playing
cards. What is the probability that at least \(1\) face card is drawn?

We can calculate the probability of being dealt at least one face card
by calculating the probability of being dealt no face cards and then
subtracting that result from 1.

In a standard deck of cards, there are \(12\) face cards and \(40\)
non-face cards.

To be dealt \(n\) non-face cards, \(n\) cards must be chosen from the
\(40\) non-face cards. There are \[{40 \choose n}\] ways to do this.
Since there are \[{52 \choose n}\] ways to choose \(n\) cards from the
entire deck, the probability of being dealt no face cards is
\[{40 \choose n} \over {52 \choose n}\] and, hence, the probability of
being dealt at least one face card is
\[1 - {{40 \choose n} \over {52 \choose n}} = 1 - {40\cdot 39 \cdots (40-n+1) \over 52\cdot 51 \cdots (52-n+1)}.\]
This result holds for \(n\leq 40\); otherwise the probability is \(0\).

\section{Conditional Probability}\label{conditional-probability}

Consider the dice-rolling experiment discussed in Example
\hyperref[dice1]{{[}dice1{]}}: two dice are rolled, one at a time. The
sample space of the experiment, \(\Omega\), has \(36\) elements,
\[\Omega =  \left\{ (1, 1), (1, 2), \ldots,  (1, 6), (2, 1), \ldots, (2, 6), \ldots, (6, 1), \ldots, (6, 6) \right\}\]
and each element of \(\Omega\) is equally likely.

Let \(A\) denote the event result of the experiment includes at least
\(1\) six; then
\[A = \left\{ (1, 6), (2, 6), (3, 6), (4, 6), (5, 6), (6, 1), (6, 2), (6, 3), (6, 4), (6, 5), (6, 6) \right\},\]
which has \(11\) elements. Hence, \(\mbox{P}(A)  = 11/36\).

Now suppose that we know that the sum of the dice is at least \(10\).
Based on this information, what is the probability that the result
includes at least \(1\) six?

Note that this probability cannot be described in terms of a single
event, because it includes the condition that the sum of the dice is at
least \(10\). It is an example of a \emph{conditional probability}.

Let \(B\) denote the event that the sum of the dice is at least \(10\);
then
\[B = \left\{ (4, 6), (5, 5), (5, 6), (6, 4), (6, 5), (6, 6) \right\}.\]
We want to find \(\mbox{P}(A\, |\, B)\), read as the ``conditional
probability of \(A\) given \(B\)". It is the probability that we roll at
least \(1\) six \textbf{given that} the sum of the dice is at least
\(10\).

There are \(6\) elements in \(B\); for \(5\) of these, there is at least
one six. Therefore, it is reasonable to expect that
\(\mbox{P}(A\, |\, B) = 5/6\).

This is, in fact, correct. The general formula for a conditional
probability is
\[\mbox{P}(A\, |\, B) = \frac{\mbox{P}(A \cap B)}{\mbox{P}(B)},\]
provided that \(\mbox{P}(B) > 0\). Note that \(A\cap B\) represents the
part of \(A\) that satisfies the condition \(B\).

In the example,
\[A \cap B = \left\{ (4, 6), (5, 6), (6, 4), (6, 5), (6, 6) \right\}\]
so that \(\mbox{P}(A \cap B)  = 5/36\). Using the fact that
\(\mbox{P}(B) = 6/36\) yields the answer given above.

Conditional probabilities are useful because they allow us to
incorporate additional information into the probability calculation.

Note that, for a given event \(B\) with \(\mbox{P}(B)>0\), the function
\(Q\) defined on subsets of \(\Omega\) and given by
\[\mbox{Q}(A) = \mbox{P}(A\, | \, B)\] is a probability function on
\(\Omega\), in the sense that it satisfies all the properties of a
probability function, such as
\[\mbox{P}(A^c\, | \, B) = 1 - \mbox{P}(A\, | \, B)\] and
\[\mbox{P}(A_1 \cup A_2\, |\, B) = \mbox{P}(A_1\, |\, B) + \mbox{P}(A_2\, |\, B) - \mbox{P}(A_1 \cap A_2\, | \, B).\]

Consider an urn with \(2\) red balls and \(w\) white balls for some
\(w\geq 2\). Suppose that \(2\) balls are randomly selected from the
urn. Given that the balls are the same color, what is the probability
that they are red?

Define two events, \(A\), the event that both balls are red and \(B\),
the event that the balls are the same color. Hence, we want to determine
\(\mbox{P}(A\, |\, B)\).

There are \[{2 + w \choose 2}\] ways to choose \(2\) balls from the urn,
which contains \(2+w\) balls. There is \(1\) way to choose \(2\) red
balls and \[{w \choose 2}\] ways to choose \(2\) white balls. Hence,
\[\mbox{P}(B) = \frac{1 + {w \choose 2}}{{2 + w \choose 2}} = \frac{w(w-1) + 2}{(w+2)(w+1)}\]
and
\[\mbox{P}(B) = \frac{1}{{2 + w \choose 2}} = \frac{2}{(w+2)(w+1)}.\]

Note that, because \(A\subset B\), \(A \cup B = A\). It follows that
\[\begin{aligned}
\mbox{P}(A\, |\, B) &= \frac{\mbox{P}(A \cap B)}{\mbox{P}(B)} = \frac{\mbox{P}(A)}{\mbox{P}(B)} \\
&= \frac{ \frac{2}{(w+2)(w+1)} }{ \frac{w(w-1) + 2}{(w+2)(w+1)}} \\
&= \frac{2}{w(w-1) + 2}. 
\end{aligned}\]

\subsection*{Multiplication law}\label{multiplication-law}
\addcontentsline{toc}{subsection}{Multiplication law}

Rewriting the expression for conditional probability yields the
\emph{multiplication law} for probabilities: for events \(A\), \(B\),
\[\mbox{P}(A \cap B) = \mbox{P}(B\, | \, A)\mbox{P}(A) = \mbox{P}(A\, | \, B)\mbox{P}(B).\]

\phantomsection\label{urn1}{} Consider an urn with \(r\) red balls and
\(b\) black balls, where \(r\) and \(b\) are positive integers. Suppose
that \(2\) balls are randomly selected from the urn, one at a time. What
is the probability that the first ball is red and the second ball is
black?

Define two events, \(A\), the event that the first ball is red and
\(B\), the event that the second ball is black. We want
\(\mbox{P}(A \cap B)\).

This is a case in which the expression
\(\mbox{P}(B\, |\, A)\mbox{P}(A)\) may be a convenient way to calculate
\(\mbox{P}(A \cap B)\): the probability that the first ball is red is
easy to determine, \[\mbox{P}(A) = \frac{r}{r + b},\] and, given the
result on the first ball, the probability that the second ball is black
is also easy to determine.

Specifically, if the first ball is red, that leaves \(r-1\) red balls
and \(b\) black balls in the urn. Hence,
\[\mbox{P}(B\, | \, A) = \frac{b}{r -1 + b}.\] It follows that
\[\mbox{P}(A\cap B) = \frac{r}{r+b}\, \frac{b}{r-1 + b}.\]

\subsection*{Independent events}\label{independent-events}
\addcontentsline{toc}{subsection}{Independent events}

Roughly speaking, events \(A\) and \(B\) are said to be
\emph{independent} if the occurrence of one event does not affect the
probability of the other, in the sense that
\[\mbox{P}(A\, | \, B) = \mbox{P}(A)\  \text{ and } \ \mbox{P}(B\, | \, A) = \mbox{P}(B).\]
Using the multiplication law, these can be written
\[\mbox{P}(A \cap B)= \mbox{P}(A) \mbox{P}(B),\] which is taken as the
definition of independence.

Consider the example of rolling two fair dice, one at a time. Let \(A\)
denote the event that result includes at least \(1\) six and \(B\)
denote the event that the sum is at least \(10\). Then, as we have seen
in the example at the beginning of this section,
\(\mbox{P}(A) = 11/36\), \(\mbox{P}(B) = 6/36\), and
\(\mbox{P}(A \cap B) = 5/36\). Thus, because
\[\frac{5}{36}\neq \frac{11}{36} \frac{6}{36},\] \(A\) and \(B\) are not
independent events. That is, knowing that the sum is of the dice is at
least \(10\) affects the probability that the result includes at least
\(1\) six. Or, alternatively, knowing that the result includes at least
\(1\) six affects the probablity that the sum is at least \(10\).

Now consider a third event, \(C\), which denotes the event that the
first die is a \(4\):
\[C = \left\{ (4, 1), (4, 2), (4, 3), (4, 4), (4, 5), (4, 6) \right).\]
Hence \(\mbox{P}(C) = 1/6\).

Note that
\[A \cap C = \left\{ (4, 6) \right\} \ \ \text{ and } \ \ B \cap C = \left\{ (4, 6) \right\}.\]
Hence, both \(\mbox{P}(A \cap C)\) and \(\mbox{P}(B \cap C)\) are
\(1/36\). Because \[\frac{1}{36} \neq \frac{11}{36} \frac{1}{6},\] it
follows that \(A\) and \(C\) are not independent. However,
\[\mbox{P}(B \cap C) = \frac{1}{36} = \frac{6}{36} \frac{1}{6} = \mbox{P}(B)\mbox{P}(C)\]
so that \(B\) and \(C\) are independent events.

Thus, knowing that the first die is a \(4\) does not affect the
probability that the sum is at least \(10\). Note that \(1/6\) of the
results in which the first die is a \(4\) have a sum of at least \(10\)
and \(1/6\) of all results have a sum of at least \(10\).

On the other hand, knowing that the first die is a \(4\) changes the
probability that the result includes at least one \(6\) from the
unconditional probability of \(11/36\) to the conditional probability of
\[\mbox{P}(A\, |\, C)  = \frac{\mbox{P}(A \cap C)}{\mbox{P}(C)} = \frac{1/36}{1/6} = \frac{1}{6}.\]

The concept of independence can be extended to an arbitrary number of
events. First consider three events, \(A_1, A_2, A_3\). These events are
said to be independent if
\[\mbox{P}(A_1 \cap A_2 \cap A_3) = \mbox{P}(A_1) \mbox{P}(A_2) \mbox{P}(A_3), \ \ P(A_1 \cap A_2) = \mbox{P}(A_1)\mbox{P}(A_2),\]
\[\ \ P(A_1 \cap A_3) = \mbox{P}(A_1) \mbox{P}(A_3) 
\ \ \text{ and } \ \ P(A_2 \cap A_3) = \mbox{P}(A_2) \mbox{P}(A_3).\]

More generally, a set of \(n\) events \(A_1, A_2, \ldots A_n\) is
independent if the probability of the intersection of any subset of the
events is the product of the probabilities of the events in the subset.
That is, for any integer \(k\), \(1 \leq k \leq n\) and indices
\(1 \leq i_1 < i_2 < \cdots < i_k \leq n\),
\[\mbox{P}(A_{i_1} \cap A_{i_2} \cap \cdots \cap A_{i_k}) = \mbox{P}(A_{i_1}) \cdots \mbox{P}(A_{i_k}).\]

(Bernoulli trials) Consider an experiment with a sample space consisting
of two basic outcomes, \(\omega_1\) and \(\omega_2\). Then there are
four possible events, \(\emptyset\),
\(\Omega= \{ \omega_1, \omega_2 \}\), \(\{\omega_1 \}\), and
\(\{\omega_2 \}\). Thus, the probability function can be described by a
single number, \(\mbox{P}(\{ \omega_1 \})\); it follows that
\(\mbox{P}(\{ \omega_2 \})  = 1 - \mbox{P}(\{ \omega_1 \})\). An
experiment of this form is known as a \emph{Bernoulli trial}.

Now consider an experiment consisting of \(n\) independent replications
of the experiment. The sample space for this second experiment is given
by \[\Omega_n = \Omega \times \Omega \times \cdots \times \Omega.\]
Thus, an element of \(\Omega_n\) can be written
\((\omega_{i_1}, \omega_{i_2}, \ldots \omega_{i_n})\), where
\(i_1, i_2, \ldots, i_n\) each take values in the set \(\{1, 2 \}\).

Let \(\mbox{P}_n\) denote the probability function of the experiment.
The term ``independent replications" refers to the fact that
\[\mbox{P}_n\left(\{ (\omega_{i_1}, \ldots, \omega_{i_n}) \} \right) =
\mbox{P}( \{ \omega_{i_1} \}) \mbox{P}(\{ \omega_{i_2} \}) \cdots \mbox{P}( \{ \omega_{i_n} \}).\]
The experiment with sample space \(\Omega_n\) and probability function
\(\mbox{P}_n\) is known as a \emph{sequence of Bernoulli trials}.

This is a generalization of the scenario considered in Example
\hyperref[binom_ex]{{[}binom\_ex{]}}, in which \(n=2\) and \(\omega_1\)
and \(\omega_2\) were denoted by \(0\) and \(1\), respectively.

\subsection*{The partition theorem}\label{the-partition-theorem}
\addcontentsline{toc}{subsection}{The partition theorem}

Consider an experiment with sample space \(\Omega\). A \emph{partition}
of \(\Omega\) is a collection of disjoint subsets of \(\Omega\),
\(\{ A_1, A_2, \ldots \}\) such that
\[\bigcup_i A_i \equiv A_1 \cup A_2 \cup \cdots = \Omega.\] Such a
partition can be either finite or infinite.

Note that any set \(B\subset \Omega\) can be written
\[B = B\cap \Omega = B \cap \left( \bigcup_i A_i \right) = \bigcup_i B\cap A_i ;\]
furthermore, \[B\cap A_1,\,  B\cap A_2, \,  \ldots\] are disjoint. It
follows that \[\label{part1}
 \mbox{P}(B) = \sum_i \mbox{P}(B \cap A_i).\]

Furthermore, if \(\mbox{P}(A_i) > 0\) for all \(i\), then
\[\label{part2}
\mbox{P}(B) = \sum_i \mbox{P}(B\, |\, A_i) \mbox{P}(A_i).\]

The results given in \hyperref[part1]{{[}part1{]}} and
\hyperref[part2]{{[}part2{]}} are known as the \emph{partition theorem};
the term \emph{law of total probability} is also used.

Consider an urn with \(r\) red balls and \(b\) black balls. Suppose that
\(2\) balls are randomly selected from the urn, one at a time. What is
the probability that the second ball is black?

We analyzed this experiment in Example \hyperref[urn1]{{[}urn1{]}}.
There we saw that the probability that the second ball is black is easy
to calculate if we know the result of the first ball. This suggests that
it may be convenient to use the partition theorem.

Define two events, \(A\), the event that the first ball is red and
\(B\), the event that the second ball is black. We have seen that
\[\mbox{P}(A) = \frac{r}{r + b} \ \text{ and } \ \  \mbox{P}(B\, | \, A) = \frac{b}{r -1 + b}.\]
The same basic argument can be used to find \(\mbox{P}(B\, |\, A^c)\):
if \(A^c\) occurs, that is, if the first ball is black, then, when the
second ball is chosen there are \(r\) red balls and \(b-1\) black balls
in the urn so that \[\mbox{P}(B\, |\, A^c) = \frac{b-1}{r+b-1}.\]

Hence, we can use the partition theorem with \(A_1 = A\) and
\(A_2 = A^c\): \[\begin{aligned}
\mbox{P}(B) &= \mbox{P}(B\, |\, A) \mbox{P}(A) + \mbox{P}(B\, |\, A^c)\mbox{P}(A^c) \\ &= \frac{b}{r-1+b}\, \frac{r}{r+b} + \frac{b-1}{r-1+b}\, \frac{b}{r+b} \\
&=\frac{br + (b-1)b}{(r+b)(r+b-1)} \\
&= \frac{b}{r+b}. 
\end{aligned}\]

\subsection*{Bayes' Theorem}\label{bayes-theorem}
\addcontentsline{toc}{subsection}{Bayes' Theorem}

Consider an experiment and let \(A, B\) be events. In some cases,
information is available regarding \(\mbox{P}(A\, |\, B)\) but we are
interested in \(\mbox{P}(B\, |\, A)\). Fortunately, the two conditional
probabilities are related through \(\mbox{P}(A\cap B)\):
\[\mbox{P}(A\cap B) =  \mbox{P}(A\, |\, B) \mbox{P}(B)  = \mbox{P}(B\, |\, A)  \mbox{P}(A).\]
It follows that
\[\mbox{P}(B\, |\, A)  = \frac{\mbox{P}(A\, |\, B) \mbox{P}(B)}{\mbox{P}(A)},\]
provided that \(\mbox{P}(A)>0\). This result is known as \emph{Bayes'
Theorem}.

Bayes' Theorem is often used in conjuction with the partition theorem so
that, for example,
\[\mbox{P}(B\, |\, A)  = \frac{\mbox{P}(A\, |\, B) \mbox{P}(B)}{\mbox{P}(A\, |\, B)\mbox{P}(B) + \mbox{P}(A\, |\, B^c)\mbox{P}(B^c)}.\]
More generally, if \(B_1, B_2, \ldots,\) is a partition of \(\Omega\),
then
\[\mbox{P}(B_j\, |\, A)  = \frac{\mbox{P}(A\, |\, B_j) \mbox{P}(B_j)}{\sum_i \mbox{P}(A\, |\, B_i)\mbox{P}(B_i)}.\]

A classic example of the use of Bayes' Theorem is in the analysis of a
diagnostic test.

Consider a medical diagnostic test for some specified disease. Suppose
it is known that

\begin{enumerate}
\def\labelenumi{\arabic{enumi}.}
\item
  \(5\%\) of all patients who take the test have the disease
\item
  the \emph{specificity} of the test is \(0.99\). That is, a patient
  known to not have the disease has a \(99\%\) chance of a negative test
\item
  the \emph{sensitivity} of the test is \(0.98\). That is, a patient
  known to have the disease has a \(98\%\) chance of a positive test
\end{enumerate}

If a particular patient has a positive result on the test, what is the
probability that the patient has the disease?

Define two events, \(A\), the event that the patient has the disease and
\(B\), the event that the patient's test is positive. The facts (1) -
(3) given above can be rewritten in probability notation:

\begin{enumerate}
\def\labelenumi{\arabic{enumi}.}
\item
  \(\mbox{P}(A) = 0.05\)
\item
  \(\mbox{P}(B^c \, | \, A^c) = 0.99\)
\item
  \(\mbox{P}(B\, | \, A) = 0.98\)
\end{enumerate}

We want to find \(\mbox{P}(A\, | \, B)\).

Using Bayes' Theorem, together with the partition theorem,
\[\begin{aligned}
\mbox{P}(A\, |\, B)  &= \frac{\mbox{P}(B\, |\, A) \mbox{P}(A)}{\mbox{P}(B\, |\, A)\mbox{P}(A) + \mbox{P}(B\, |\, A^c)\mbox{P}(A^c)} \\
&= \frac{ (0.98)(0.05)}{(0.98)(0.05) + (0.01)(0.95)} \\
&= 0.838.
\end{aligned}\]

Consider an urn with \(3\) red balls and \(2\) black balls and consider
the experiment in which two balls are chosen from the urn without
replacement. Let \(A\) denote the event that the first ball selected is
a red ball and let \(B\) denote the event that the second ball selected
is a red ball. Find \(\mbox{P}(A\, |\, B)\).

In this example, probabilities that are conditonal on \(A\), that is,
conditional on the outcome of the first ball, are relatively easy to
calculate. For instance, if the first ball is red, then the probability
that the second ball is red is \(2/4 = 1/2\); if the first ball is
black, then the probability that the second ball is red is \(3/4\).
Thus,
\[\mbox{P}(B\, |\, A) = 1/2 \ \ \text{ and } \ \ \mbox{P}(B\, |\, A^c) = 3/4\]
and, using the fact that \(\mbox{P}(A) = 3/5\), it follows from the law
of total probability that
\[\mbox{P}(B) = \mbox{P}(B\, |\, A) \mbox{P}(A) + \mbox{P}(B\, |\, A^c)\mbox{P}(A^c) = \frac{1}{2}\frac{3}{5} + \frac{3}{4}\frac{2}{5} = \frac{3}{5}.\]

To find \(\mbox{P}(A\, |\, B)\) we can use Bayes' Theorem:
\[\mbox{P}(A\, |\, B) = \frac{\mbox{P}(B\, |\, A) \mbox{P}(A)}{\mbox{P}(B)} = \frac{(1/2)(3/5)}{3/5} = \frac{1}{2}.\]

\bookmarksetup{startatroot}

\chapter{Random Variables}\label{sec-chap2}

\newcommand{\cip}{\xrightarrow[]{p}}
\renewcommand{\P}{\mbox{P}}
\newcommand{\E}{\mbox{E}}
\newcommand{\Fhat}{\hat{F}}
\newcommand{\intii}{\int_{-\infty}^\infty}
\newcommand{\Ybar}{\bar{Y}}
\newcommand{\Var}{\mbox{Var}}
\newcommand{\lam}{\lambda}
\newcommand{\Ph}{\widehat{\mbox{P}}}
\newcommand{\I}{\mbox{I}}
\newcommand{\cid}{\xrightarrow[]{D}}

\newcommand{\bfmu}{\bm{\mu}}  
\newcommand{\bfw}{\bm{w}}
\newcommand{\bfR}{\bm{R}}
\newcommand{\bfone}{\bm{1}}
\newcommand{\bfx}{\bm{x}}
\newcommand{\bfy}{\bm{y}}
\newcommand{\bfz}{\bm{z}}
\newcommand{\bfV}{\bm{\mathcal{V}}}
\newcommand{\bfVm}{\bm{V}}
\newcommand{\bfq}{\bm{q}}
\newcommand{\bfC}{\bm{C}}
\newcommand{\bfI}{\bm{I}}
\newcommand{\bfdelta}{\bm{\delta}}
\newcommand{\bfbeta}{\bm{\beta}}
\newcommand{\bfS}{\bm{S}}
\newcommand{\bfalpha}{\bm{\alpha}}
\newcommand{\bfepsilon}{\bm{\epsilonilon}}
\newcommand{\bfd}{\bm{d}}
\newcommand{\bfSigma}{\bm{\Sigma}}
\newcommand{\tbfSigma}{\tilde{\bfSigma}}
\newcommand{\bfzero}{\bm{0}}
\newcommand{\bfA}{\bm{A}}
\newcommand{\bfa}{\bm{a}}
\newcommand{\bfB}{\bm{B}}
\newcommand{\bfh}{\bm{h}}
\newcommand{\bfv}{\bm{v}}
\newcommand{\bfX}{\bm{X}}
\newcommand{\bfu}{\bm{u}}
\newcommand{\bfg}{\bm{g}}
\newcommand{\bfb}{\bm{b}}
\newcommand{\muhat}{\hat{\mu}}
\newcommand{\bfM}{\bm{M}}
\newcommand{\bfGamma}{\bm{\Gamma}}

\newcommand{\p}{{q}}
\newcommand{\bfF}{\bm{F}}
\newcommand{\phat}{\hat{p}}
\newcommand{\that}{\hat{\theta}}
\newcommand{\bbar}{\bar{b}}
\newcommand{\mhat}{\hat{m}}
\newcommand{\knumer}{\frac{1}{nh} \sum_{j=1}^n K(\frac{x - X_j}{h})Y_j }
\newcommand{\given}{{\ | \ }}
\newcommand{\fhat}{\hat{f}}
\newcommand{\Cov}{\mbox{Cov}}
\newcommand{\dmX}{\mathbb{{X}}}
\newcommand{\covS}{\mathbb{S}}
\newcommand{\bfe}{\bm{e}}
\newcommand{\bfell}{\bm{\ell}}
\newcommand{\bfY}{\bm{Y}}
\newcommand{\bfL}{\bm{L}}
\newcommand{\bfPsi}{\bm{\Psi}}
\newcommand{\bfQ}{\bm{Q}}
\newcommand{\bfT}{\bm{T}}
\newcommand{\X}{\mathcal{X}}
\newcommand{\Y}{\mathcal{Y}}
\newcommand{\xbar}{\bar{x}}

\newcommand{\theto}{\theta_0}

\newcommand{\bfZ}{\bm{Z}}
\newcommand{\bfP}{\bm{P}}
\newcommand{\bfD}{\bm{D}}
\newcommand{\bfO}{\bm{O}}
\newcommand{\bfLambda}{\bm{\Lambda}}
\newcommand{\bfc}{\bm{c}}
\newcommand{\bfE}{\bm{E}}
\newcommand{\bfU}{\bm{U}}
\newcommand{\bff}{\bm{f}}
\newcommand{\bfH}{\bm{H}}
\newcommand{\Xbar}{\bar{X}}

\newcommand{\ybar}{\bar{y}}

\newcommand{\calP}{\mathcal{P}}
\newcommand{\calY}{\mathcal{Y}}
\newcommand{\calX}{\mathcal{X}}
\newcommand{\calR}{\mathcal{R}}
\newcommand\BB{\rule[-1ex]{0pt}{0pt}}  
\newcommand{\psihat}{{\widehat{\psi}}}
\newcommand{\thetast}{\theta^*}
\newcommand{\thetst}{\theta^*}

\newcommand\T{\rule{0pt}{2.6ex}}       
\newcommand\B{\rule[-1ex]{0pt}{0pt}}

\section{Introduction}\label{introduction}

Consider an experiment with sample space \(\Omega\) and let \(\omega\)
denote an outcome of the experiment so that \(\omega\in\Omega\). In many
applications we are concerned primarily with certain numerical
characteristics of \(\omega\), rather than with \(\omega\) itself. A
\emph{random variable} is a function that takes an element of \(\Omega\)
and returns a real number.

More formally, define a function \(X: \Omega\to \mathcal{X}\), where
\(\mathcal{X}\) is a subset of \(\Re\), denote a random variable; the
set \(\mathcal{X}\) is called the \emph{range} of \(X\) or, sometimes,
the \emph{sample space} of \(X\). For a given outcome
\(\omega\in \Omega\), the corresponding value of \(X\) is
\(x=X(\omega)\).

Probabilities regarding \(X\) may be obtained from the probability
function \(\mbox{P}\) for the original experiment. Let \(\mbox{P}_X\)
denote a function such that for any set \(A\subset \mathcal{X}\),
\(\mbox{P}_X(A)\) denotes the probability that \(X\in A\). Then
\(\mbox{P}_X\) is a probability function defined on subsets of
\(\mathcal{X}\) and
\[\mbox{P}_X(A) = \mbox{P}\big(\{ \omega\in\Omega: X(\omega)\in A \}\big).\]
Here \[\{ \omega\in\Omega: X(\omega)\in A \}\] is the set of all basic
outcomes \(\omega\) such that the corresponding value of the random
variable \(X\), \(X(\omega)\), is in the set \(A\).

Note that, because \(P_X\) defines a probability function on the subsets
of \(\mathcal{X}\), it must satisfy conditions (P1) - (P3). Also, it is
often convenient to proceed as if probability function \(\mbox{P}_X\) is
defined on the entire space \(\Re\). Then the probability of any subset
of \(\mathcal{X}^c\) is \(0\) and, for any set \(A \subset \Re\),
\[\mbox{P}_X(A) \equiv \Pr(X\in A) = \Pr(X \in A\cap \mathcal{X}) .\]

We will generally use a less formal notation in which \(\Pr(X\in A)\)
denotes \(\mbox{P}_X(A)\). For instance, the probability that
\(X\leq 1\) may be written as either \(\Pr(X\leq 1)\) or
\(\mbox{P}_X\left( (-\infty, 1]\right)\).

Consider an experiment with two possible outcomes,
\(\omega_1, \omega_2\); recall that such an experiment is called a
Bernoulli trial. Hence, \(\Omega = \{ \omega_1, \omega_2\}\).

Define a function \(X\) on a \(\Omega\), i.e., a random variable, by
\[X(\omega) = \begin{cases} 1 & \text{ if } \omega = \omega_1 \\
                                          0 & \text{ if } \omega = \omega_2. 
\end{cases}\] Thus, the range of \(X\) is \(\{0, 1 \}\).

Let \(\theta = \mbox{P}(\{ \omega_1 \})\). Then
\[\Pr(X = 1) = \theta \ \ \text{ and } \ \ \Pr(X = 0) = 1 - \theta.\]

A random variable with these properties is said to be a \emph{Bernoulli
random variable}.

\phantomsection\label{binom_ex2}{}

Consider an experiment with sample space
\[\Omega = \{ x\in \Re^n: x = (x_1, \ldots, x_n), x_j = 0 \ \hbox{ or } 1, \ \ 
j=1, \ldots, n \}\] so that an element of \(\Omega\) is a vector of ones
and zeros.

For \(\omega = (x_1, \ldots, x_n) \in \Omega\), take
\[\mbox{P}(\omega)  = \prod_{j=1}^n \theta^{x_j} (1-\theta)^{1-x_j}\]
where \(0 < \theta < 1\) is a given constant.

For an element \(\omega\in\Omega\), define
\[X(\omega) = \sum_{j=1}^n x_j\] so that \(X(\omega)\) is the number of
ones in \(\omega\).

Then \[\Pr(X = 0) = \mbox{P}((0, 0, \ldots, 0)) = (1-\theta)^n,\]
\[\Pr(X = 1) = \mbox{P}((1, 0, \ldots, 0)) + \mbox{P}((0, 1, 0, \ldots, 0)) + \cdots 
+ \mbox{P}((0, 0, \ldots, 0, 1)) = n \theta (1-\theta)^{n-1}.\] More
generally,
\[\Pr(X = x) = {n \choose x} \theta^x (1-\theta)^{n-x}, \ \ x=0, 1, \ldots, n;\]
to show this, note that each basic outcome
\(\omega = (x_1, \ldots, x_n)\) such that \(\sum_{j=1}^n x_j = x\) has
probability \(\theta^x (1-\theta)^{n-x}\) and there are
\[{n \choose x}\] basic outcomes with \(\sum_{j=1}^n x_j = x\).

\(X\) is said to have a \emph{binomial distribution} with parameters
\(n\) and \(\theta\).

Consider the experiment with sample space \(\Omega = (0, 1)\) and
probability function \(\mbox{P}(\cdot)\) such that
\[\mbox{P}(A) = \int_A dx, \ \ A\subset (0, 1).\]

Define a random variable \(X\) by
\[X(\omega) = \omega, \ \ 0 < \omega < 1.\] Then, for any
\(A\subset (0, 1)\), \[\Pr(X\in A) = \int_A dt.\]

As discussed above, we may take the range of \(X\) to be \(\Re\). Then,
for any subset \(A\in \Re\), \[\Pr(X\in A) = \int_{A \cap (0, 1)} dt.\]
For instance, \[\Pr(0.6 < X < 2) = \int_{0.6}^1 dx = 0.4.\]

Note that, in the previous example, we could have simply defined the
random variable \(X\) to have probability function given by
\[\Pr(X\in A) = \int_{A \cap (0, 1)} dt, \ \ A\subset \Re\] without
referring to an underlying experiment.

In this course, we will usually define random variables in that way --
without first constructing an experiment. However, it is sometimes
important to keep in mind that a random variable is a real-valued
function defined on the sample space \(\Omega\) of some experiment.

\section{Distribution Functions}\label{distribution-functions}

Consider a random variable \(X\). The properties of \(X\) are described
by its probability function \(\mbox{P}_X\), which gives the probability
that \(X\in A\) for any set \(A\subset \Re\).

However, it is often convenient to specify the distribution of a random
variable by considering \(\Pr(X\in A)\) for a more limited class of sets
\(A\). That is, we do not need to specify \(\Pr(X \in A)\) for
\textbf{all} subsets of \(\Re\) in order to describe the properties of
\(X\); using a smaller class of subsets of \(\Re\) in this context leads
to a simplification of the probability theory for random variables.

For instance, consider sets of the form \((-\infty, x]\), for
\(x\in \Re\), so that \[\Pr(X \in (-\infty, x]) = \Pr(X \leq x).\] The
\emph{distribution function} of the distribution of \(X\) or, simply,
the distribution function of \(X\), is the function
\(F\equiv F_X: \Re\to [0, 1]\) given by
\[F(x) = \Pr(X \leq x), \ \ -\infty < x < \infty.\]

\phantomsection\label{uniform_ex}{}

Suppose that \(X\) is a random variable such that
\[\Pr(X\in A) = \int_{A\cap (0, 1)}  \ dx, \ \ A\subset \Re;\] \(X\) is
said to have a uniform distribution on \((0, 1)\).

The distribution function of this distribution is given by
\[F(x) = \Pr\{ X \in (-\infty, x] \}=\int_{(-\infty, x]\cap (0, 1)} dt = 
\begin{cases}0 & \text{ if } x \leq 0 \\
       x & \text{ if } 0 < x \leq 1 \\
       1 & \text{ if } x > 1 .
\end{cases}\] Figure \hyperref[uni_plot]{1.1} gives a plot of \(F\).

\begin{figure}

{\centering \includegraphics[width=0.75\linewidth,height=\textheight,keepaspectratio]{fig2.383.pdf}

}

\caption{Distribution Function of the Uniform Distribution on
\((0, 1)\)}

\end{figure}%

\phantomsection\label{binom_ex3}{}

Let \(X\) denote a random variable with a binomial distribution with
parameters \(n\) and \(\theta\), as described in Example
\hyperref[binom_ex2]{{[}binom\_ex2{]}}. Then
\[\Pr(X = x) = {n\choose x} \theta^x (1-\theta)^{n-x}, \ \ x=0, 1, \ldots, n\]
and, hence, the distribution function of \(X\) is
\[F(x) = \sum_{j=0, 1, \ldots; j\leq x} {n \choose j} \theta^j 
(1-\theta)^{n-j}.\] Thus, \(F\) is a step function, with jumps at
\(0, 1, 2, \ldots, n\).

For instance, suppose that \(X\) has a binomial distribution with
parameters \(n=2\) and \(\theta = 1/4\). Then
\[\Pr(X = 0) ={2 \choose 0} \left(\frac{1}{4}\right)^0 \left(\frac{3}{4}\right)^2 = \frac{9}{16},\]
\[\Pr(X = 1) = {2 \choose 1} \left(\frac{1}{4}\right)^1 \left(\frac{3}{4}\right)^1 = \frac{6}{16},\]
and, by subtraction, \(\Pr(X = 2) = 1/16\).

It follows that, for \(x < 0\), \(\Pr(X \leq x) = 0\), for
\(0\leq x < 1\), \[\Pr(X \leq x) = \Pr(X = 0) = \frac{9}{16},\] for
\(1 \leq x < 2\),
\[\Pr(X \leq x) = \Pr(X = 0 \cup X=1) = \Pr(X=0) + \Pr(X = 1) = \frac{15}{16},\]
and for \(x\geq 2\), \(\Pr(X \leq x) = 1\).

Hence, the distribution function of \(X\) is given by \[F(x) =  
\begin{cases}0 & \text{ if } x < 0 \\
       \frac{9}{16} & \text{ if } 0 \leq x < 1 \\
       \frac{15}{16} & \text{ if } 1 \leq x < 2 \\
       1 & \text{ if } x \geq 2 .
\end{cases}\] Figure \hyperref[binom_plot]{1.2} gives a plot of \(F\).

\begin{figure}

{\centering \includegraphics[width=0.75\linewidth,height=\textheight,keepaspectratio]{fig1.383.pdf}

}

\caption{Distribution Function of the Binomial Distribution with
Parameters \(n=2\) and \(\theta = 1/4\)}

\end{figure}%

Clearly, there are some basic properties which any distribution function
\(F\) must possess; these are given as properties (DF1)--(DF3) below.
Furthermore, if a function \(F:\Re\mapsto [0, 1]\) satisfies
(DF1)--(DF3), then there exists a random variable \(X\) such that \(F\)
is the distribution function of \(X\).

\begin{enumerate}
\def\labelenumi{\arabic{enumi}.}
\item
  \(\lim_{x\to\infty} F(x) = 1\); \(\lim_{x\to -\infty} F(x) = 0\)
\item
  If \(x_1 < x_2\) then \(F(x_1) \leq F(x_2)\)
\item
  \(\lim_{h\to 0^+} F(x+h) = F(x)\)
\end{enumerate}

To see why (DF1) must hold, note that, for any \(x\), \(F(x)\) is a
probability, \(F\) must take values in \([0, 1]\) and, as \(x\)
approaches \(\infty\),
\[\Pr(X \leq x) \ \ \text{ approaches } \ \ \Pr(X < \infty)  = 1.\]
Similarly, \(\Pr(X \leq x)\) must approach \(0\) as \(x\to-\infty\).

For \(h>0\) \[\begin{aligned}
F(x+h)&= 
\Pr(X \leq x+h)\\ &= \Pr(X \leq x \cup x < X \leq x+h)\\
&= \Pr(X \leq x) + \Pr(x < X \leq x+h) \\
&= F(x) + \Pr(x < X \leq x+h) \\
\geq F(x); 
\end{aligned}\] hence, \(F\) must be a nondecreasing function, as stated
in (DF2).

A distribution function is not necessarily a continuous function;
however, according to (DF3) (which will not be proven here), a
distribution function is \emph{right-continuous}. Note that the notation
\(h\to 0^+\) refers to a sequence of positive numbers that approaches
\(0\); hence, \(x+h > x\) and \(x+h \to x\) as \(h\to 0^+\).

An example of the right-continuity property of distribution functions is
given in Figure \hyperref[binom_plot]{1.2}. Clearly, the distribution
function in that figure is not continuous -- there are jumps at
\(x=0, 1\), and \(2\). However, it is right-continuous at those points.
For instance, consider the limit of \(F(x)\) as \(x\) approaches \(2\)
from above. For all \(x \geq 2\), \(F(x) = 1\) so that
\[\lim_{h\to 0^+} F(2+h) = 1 = F(2).\]

On the other hand, for all \(1\leq x< 2\), \(F(x) = 15/16\) so that as
\(x\) approaches \(2\) from below the limit is \(15/16\):
\[\lim_{h\to 0^-} F(2+h) = 15/16.\] It follows that \(F(x)\) is not
left-continuous at \(x=2\).

Let \(F_1\) and \(F_2\) denote distribution functions for some random
variables. Is the function given by \[F(x) = F_1(x) F_2(x),  \ x\in\Re\]
also a distribution function?

Recall that that \(F\) is a distribution function if it satisfies
conditions (DF1) through (DF3).

Consider \(F(x) = F_1(x) F_2(x)\). Clearly,
\[\lim_{x\to\infty} F_1(x) F_2(x) = \lim_{x\to\infty} F_1(x) \ \lim_{x\to\infty} 
F_2(x) = 1\] and
\[\lim_{x\to -\infty} F_1(x) F_2(x) = \lim_{x\to\-\infty} F_1(x) \ 
\lim_{x\to -\infty} F_2(x) = 0,\] establishing (DF1). Similarly,
\[\lim_{h\to 0^+} F_1(x+h) F_2(x+h) = \lim_{h\to 0^+} F_1(x+h) \ \lim_{h\to 0} 
F_2(x+h) = F_1(x) F_2(x),\] verifying (DF3). Because
\(F_1(x_1) \leq F_1(x_2)\) and \(F_2(x_1) \leq F_2(x_2)\) for
\(x_1 < x_2\), it follows that
\(F_1(x_1) F_2(x_1) \leq F_1(x_2) F_2(x_2)\), establishing (DF2). Hence,
\(F\) is a distribution function.

Because of properties (DF1) and (DF2), when giving the form of a
distribution function, it is convenient to only give the value of the
function in the range of \(x\) for which \(F(x)\) varies between \(0\)
and \(1\). For instance, in the uniform distribution example, in which
\[F(x) = \Pr\{ X \in (-\infty, x] \}=\int_{(-\infty, x]\cap (0, 1)} dx = 
\begin{cases}0 & \text{ if } x \leq 0 \\
       x & \text{ if } 0 < x \leq 1 \\
       1 & \text{ if } x > 1 
\end{cases}\] we could say that \(F(x) = x\), \(0 \leq
 x \leq 1\); in this case it is understood that \(F(x) = 0\) for
\(x< 0\) and \(F(x) = 1\) for \(x>1\).

A distribution function \(F\) gives the probability of sets of the form
\((-\infty, x]\). However, it can also be used to give the probability
of an bounded interval of the form \((x_1, x_2]\) where \(x_1 < x_2\).

Note that \[(-\infty, x_1] \cup (x_1, x_2] = (-\infty, x_2];\]
furthermore, \((-\infty, x_1]\) and \((x_1, x_2]\) are disjoint subsets
of \(\Re\). It follows that
\[\Pr(X \in (-\infty, x_2]) = \Pr(X \in (-\infty, x_1]) + \Pr(X \in (x_1, x_2]).\]
Because
\[\Pr(X \in (-\infty, x_2]) = F(x_2)  \ \ \text{ and } \ \  \Pr(X \in (-\infty, x_1]) = F(x_1),\]
it follows that \[\Pr(X \in (x_1, x_2]) = F(x_2) - F(x_1).\]

Thus, we have the following useful result. Let \(X\) denote a random
variable with distribution function \(F\); Then, for \(x_1 < x_2\),
\[\Pr( x_1 < X \leq x_2) = F(x_2) - F(x_1).\]

Let \(X\) denote a random variable with distribution function
\[F(x) = x^2(3 - 2x), \ \ 0 < x < 1.\]

Then
\[\Pr(0.2 < X \leq 0.5) = F(0.5) - F(0.2) = (0.5)^2(3 - 2(0.5)) - (0.2)^2(3 - 2(0.2)) = 0.3960\]
and
\[\Pr(X > 0.8) = \Pr(0.8 < X \leq \infty) = 1 - F(0.8) = 1 - (0.8)^2(3 - 2(0.8)) = 0.1040.\]

Another important property of distribution functions is that the
distribution function of a random variable completely characterizes its
distribution: two random variables with the same distribution function
have the same probability distribution.

Let \(X_1\) and \(X_2\) denote random variables and let \(F_j\) denote
the distribution function of \(X_j\), \(j=1, 2\).

If \[F_1(x) = F_2(x) \ \ \text{ for all } \ \ -\infty < x < \infty,\]
then
\[\Pr(X_1 \in A) = \Pr(X_2 \in A) \ \ \text{ for any set} \ \ A\subset \Re.\]

A proof this result requires sophisticated mathematical techniques and
is beyond the scope of this course. However, it is not difficult to give
an informal explanation of why we expect such a result to hold.

The goal is to show that, if \(X_1\) and \(X_2\) have the same
distribution function (denoted by \(F\)), then, for `any' set
\(A\subset \Re\), \[\Pr(X_1\in A) = \Pr(X_2\in A).\] First suppose that
\(A\) is an interval of the form \((a_0, a_1]\). Then
\[\Pr(X_j \in A) = F(a_1) - F(a_0), \ \ j=1, 2\] so that
\(\Pr(X_1 \in A) = \Pr(X_2 \in A)\). The same is true for \(A^c\). Now
consider a second interval \(B=(b_0, b_1]\). Then
\[A\cap B = \begin{cases} \emptyset & \text{ if } b_0 > a_1 \text{ or } a_0 > b_1 \\
                    B & \text{ if } a_0 \leq b_0 < b_1 \leq a_1 \\ 
                    A & \text{ if } b_0 \leq a_0 < a_1 \leq b_1 \\
                    (a_0, b_1] & \text{ if } b_1 \leq a_1 \text{ and } b_0 \leq a_0 \\
                    (b_0, a_1] & \text{ if }  a_1 \leq b_1 \text{ and } a_0 \leq b_0 
\end{cases}.\] In each case, \(A\cap B\) is an interval and, hence,
\(\Pr(X_j \in A\cap B)\) and \(\Pr(X_j \in A \cup B)\) do not depend on
\(j=1,2\). The same approach can be used for any finite intersection of
intervals.

Also note that, because \[(A \cup B)^c = A^c \cap B^c\] and, if
\(\Pr(X_j \in (A\cup B)^c)\) does not depend on \(j\) then
\(\Pr(X_j \in (A\cup B))\) does not depend on \(j\), the same type of
result holds for any finite union of intervals and for any finite
combination of unions and intersections of intervals.

Hence, if \(X_1\) and \(X_2\) have the same distribution function, then
\[\Pr(X_1 \in A) = \Pr(X_2 \in A)\] for any set \(A\subset \Re\) that
can be written in terms of a finite number of intervals of the form
\((a, b]\) using unions and intersections. The mathematical challenge is
to extend the result to `any' subset of \(\Re\).

\section{Discrete Distributions}\label{discrete-distributions}

Let \(X\) denote a random variable with range \(\mathcal{X}\). Suppose
that \(\mathcal{X}\) is a countable set, that is, it is either a finite
set of the form \[\mathcal{X}= \left\{ x_1, x_2, \ldots, x_m \right\}\]
or it is an infinite set of the form
\[\mathcal{X}= \left\{ x_1, x_2, \ldots, \right\}\] where
\(x_1, x_2, \ldots\) are real numbers. In this case, we say that \(X\)
has a \emph{discrete distribution} or is a \emph{discrete random
variable}. We can assume that the elements \(x_1, x_2, \ldots\) of
\(\mathcal{X}\) are ordered so that \[x_1 < x_2 < \cdots.\]

Define a function \(p:\Re\to [0, 1]\) by \[p(x) = \Pr(X = x).\] The
function \(p\) is called the \emph{mass function} of the distribution;
the term \emph{frequency function} is also used. Note that we assume
that \(p\) is defined for all \(-\infty < x < \infty\); if
\(x\notin \mathcal{X}\), then \(p(x) = 0\).

Clearly, because probabilities are nonnegative, \(p(x)\geq 0\) for all
\(x\). Furthermore, if \(\mathcal{X}\) is finite, with \(m\) elements
\[\label{mass_sum1}
 \sum_{j=1}^m p(x_j) = 1;\] if \(\mathcal{X}\) is a countably infinite
set, then \[\label{mass_sum2}
\sum_{j=1}^\infty p(x_j) = 1.\]

These results follow from the fact that (in the finite \(\mathcal{X}\)
case), we can write \[\mathcal{X}= \cup_{j=1}^m \{ x_j \}\] where the
sets \(\{x_1 \}, \{x_2 \}, \ldots\) are disjoint. Then \[\begin{aligned}
 \Pr(X \in \mathcal{X}) &= \sum_{j=1}^m \Pr(X \in \{x_j \}) = \sum_{j=1}^m \Pr(X = x_j) \\
&= \sum_{j=1}^m p(x_j).
\end{aligned}\] The result \hyperref[mass_sum1]{{[}mass\_sum1{]}} now
follows from \(\Pr(X \in \mathcal{X}) = 1\);
\hyperref[mass_sum2]{{[}mass\_sum2{]}} follows from a similar argument.

Consider \(\Pr(X \in A)\) where \(A\subset \Re\). Note that
\[\Pr(X \in A) = \Pr(X \in A \cap \mathcal{X})\] and that
\[A \cap \mathcal{X}= \{x \in \mathcal{X}: x \in A\}\] so that
\[\Pr(X \in A) = \sum_{j: x_j\in A} p(x_j).\]

In particular, the distribution function of a discrete random variable
is discontinuous, with jumps at each value in its range. Let \(X\) be a
discrete random variable with mass function \(p(\cdot)\) and range
\(\mathcal{X}\) with elements \(x_j\). Then the distribution function of
\(X\) is given by \[F(x) = \sum_{j: x_j\leq x} p(x).\]

For instance, suppose that \(x_1 \leq x < x_2\). Then \(F(x) = p(x_1)\);
however, \(F(x_2) = p(x_1)+p(x_2)\) so that \(F(x)\) has a jump at
\(x=x_2\). Thus, \(F\) is a step function with jumps at each \(x_j\). We
have already seen one instance of this property, in the binomial
distribution discussed in Example
\hyperref[binom_ex3]{{[}binom\_ex3{]}}; see Figure
\hyperref[binom_plot]{1.2}.

The converse is also true. That is, if a distribution function \(F\) is
a step function, with jumps at \(x_1, x_2, \ldots\), then the
distribution is discrete with range
\(\mathcal{X}= \{x_1, x_2, \ldots \}\) and mass function \(p(\cdot)\),
where \(p(x_j)\) is the size of the jump of \(F\) at \(x_j\).

Let \(X\) denote a random variable with range
\(\mathcal{X}= \{1, 2, \ldots, m \}\) for some \(m=1, 2, \ldots\), and
let \[\theta_j = \Pr(X = j), \ \ j=1, \ldots, m.\] Therefore,
\(\theta_j\) are nonnegative, and sum to \(1\);
\(\theta_1 + \theta_2 + \cdots + \theta_m = 1\).

The distribution function of \(X\) is given by
\[F(x) = \begin{cases} 0 & \text{ if } x < 1 \\ 
\theta_1 & \text{ if } 1 \leq x < 2 \\ 
\theta_1 + \theta_2 & \text{ if } 2 \leq x < 3 \\
\vdots \\ 
\theta_1 + \cdots + \theta_{m-1} & \text{ if } m-1 \leq x < m \\ 
1 & \text{ if } m \leq x \\
\end{cases}.\]

\phantomsection\label{binom_ex4}{}

Let \(X\) denote the random variable defined in Example
\hyperref[binom_ex3]{{[}binom\_ex3{]}}. Then
\(\mathcal{X}= \{0, 1, \ldots, n \}\) and
\[\Pr(X = x) = {n \choose x} \theta^x (1-\theta)^{n-x}, \ \ x=0, 1, \ldots, 
n;\] here \(0 < \theta < 1\) is a constant.

Then \(X\) is a discrete random variable with mass function
\[p(x) = {n \choose x} \theta^x (1-\theta)^{n-x}, \ \ x=0, 1, \ldots, n.\]

Let \(X\) denote a random variable with range
\(\mathcal{X}= \{1, 2, \ldots\}\) and mass function
\[p(x) = \left( \frac{1}{2} \right)^x, \ \ x=1, 2, \ldots.\] Note that
\[\sum_{x=1}^\infty \left( \frac{1}{2} \right)^x =1.\]

For a positive integer \(m\),
\[\Pr(X \leq m) = \sum_{x=1}^m \left( \frac{1}{2} \right)^x = \frac{1/2 - (1/2)^{m+1}}{1 - 1/2}
= 1 - \left( \frac{1}{2}\right)^{m}.\] Therefore, the distribution
function of \(X\) is given by
\[F(x) = 1 - \left(\frac{1}{2}\right)^m \ \ \text{ for } \ \ m \leq x < m+1.\]

Let \(X\) denote a random variable with a discrete distribution with
mass function
\[p_X(x) = \theta ( 1- \theta)^{x-1}, \ \ x=1, 2, \ldots,\] where
\(0 < \theta < 1\). Consider the probability that \(X\) is even, as a
function of \(\theta\).

Note that the event that \(X\) is even is given by
\(\{2, 4, \ldots \}\). It follows that the probability that \(X\) is
even is
\[\Pr\left( X\in \{2, 4, \ldots \} \right) = \sum_{x=1}^\infty p_X(2x) = 
\frac{\theta}{1-\theta} \sum_{x=1}^\infty (1 - \theta)^{2x}.\]

We know that, for \(0 < p < 1\),
\[\sum_{j=1}^\infty p^j = \frac{p}{1-p};\] it follows that
\[\sum_{x=1}^\infty (1 - \theta)^{2x} =\frac{(1- \theta)^2}{1 - (1-\theta)^2}.\]

Therefore, the probability that \(X\) is even is
\[\frac{\theta}{1-\theta} \frac{(1-\theta)^2}{1 - (1-\theta)^2} = \frac{\theta(1-\theta)}{2\theta - \theta^2} = \frac{1 - \theta}{2 - \theta}.\]

\section{Continuous Distributions}\label{den_sec}

Consider a random variable \(X\) with distribution function \(F\) and
range \(\mathcal{X}\). Suppose there exists a function \(p:\Re\to\Re\)
such that \[\label{cont_dist}
 F(x)  = \int_{-\infty}^x p(t) dt, \ \ -\infty < x < \infty .\] The
function \(p\) is called the \emph{density function} of the distribution
or, more simply, of \(X\).

Because an integral is a continuous function of its limits, the
distribution function \(F\) must be a continuous function. Hence, when
\hyperref[cont_dist]{{[}cont\_dist{]}} holds, we say that the
distribution of \(X\) is an \emph{continuous distribution};
alternatively, we say that \(X\) is an \emph{continuous random
variable}.

Because \(F\) is non-decreasing, \(p\) can be assumed to be non-negative
and because the limit of \(F(x)\) as \(x\to\infty\) is \(1\), we must
have \[\int_{-\infty}^\infty p(x) dx = 1.\]

Thus, if \(p\) is a nonnegative function satisfying
\[\int_{-\infty}^\infty p(x) dx = 1\] then \(p\) is a density function
of a continuous distribution and there exists a continuous random
variable with \(p\) as its density function. That is, a continuous
distribution can be specified by giving its density function.

\phantomsection\label{uni_ex_den}{}

Let \(X\) denote a random variable with the uniform distribution on
\((0, 1)\), as defined in Example \hyperref[uni_ex]{{[}uni\_ex{]}}. In
Example \hyperref[uniform_ex]{{[}uniform\_ex{]}} it is shown that the
distribution function of this distribution is given by
\[F(x) = \Pr\{ X \in (-\infty, x] \}=\int_{(-\infty, x]\cap (0, 1)} dt = 
\begin{cases}0 & \text{ if } x \leq 0 \\
       x & \text{ if } 0 < x \leq 1 \\
       1 & \text{ if } x > 1 .
\end{cases}\]

Define a function \(p\) by
\[p(x) = \begin{cases}1 & \text{ if } 0 \leq x \leq 1 \\
0 & \text{ otherwise} 
\end{cases};\] then
\[F(x) = \int_0^x p(t) dt, \ \ -\infty < x < \infty.\]

It follows that \(X\) has an continuous distribution with density
function \(p\).

Let \(F\) denote a distribution function for a random variable with a
continuous distributions and let \(p\) denote the corresponding density
function.

Is \(\alpha p(\alpha x)\), where \(\alpha>0\), also the density function
for a random variable with a continuous distribution?

Note that, for \(\alpha>0\), \(\alpha p(\alpha x)\) is nonnegative and
\[\int_{-\infty}^\infty \alpha p(\alpha x) dx = \int_{-\infty}^\infty p(x)dx = 1;\]
hence, \(\alpha p(\alpha x)\) is a density function.

For a random variable \(X\) with distribution function \(F\), we have
seen that \[\Pr(x_1 < X \leq x_2) = F(x_2) - F(x_1).\]

Therefore, if \(X\) has a continuous distribution with density \(p\), it
follows that the density satisfies \[\begin{aligned}
\Pr(x_1 < X \leq x_2) &= \int_{-\infty}^{x_2} p(t)\, dt - \int_{-\infty}^{x_1} p(t)\, dt \\
&= \int_{x_1}^{x_2} p(t)\, dt.
\end{aligned}\]

It is important to note that there is a slight technical complication to
the definition of a density function of a continuous distribution: it is
not uniquely defined. Consider two nonnegative functions \(p_1\) and
\(p_2\), such that \[\label{eq_den}
 p_1(x) = p_2(x) \ \ \text{ for all }\ x \text{ except } \tilde{x}_1, \tilde{x}_2, \ldots .\]
If \[F(x) = \int_{-\infty}^x p_1(t) dt, \ \ -\infty < x < \infty,\] then
\[F(x) = \int_{-\infty}^x p_2(t) dt, \ \ -\infty < x < \infty.\]

The reason for this is that changing the value of a function at a single
point (or at a few isolated points) does not change the value of the
integral. In this case, either \(p_1\) or \(p_2\) may be taken as the
density function of the distribution. Generally, we use the simplest
version of the density.

When a condition holds for all \(x\), except for \(x\) in a countable
set as in \hyperref[eq_den]{{[}eq\_den{]}}, we say that the condition
hold for \emph{almost all} \(x\). Thus, \hyperref[eq_den]{{[}eq\_den{]}}
could be written \[p_1(x) = p_2(x) \ \ \text{ for almost all }  \ x .\]

In spite of the nonuniqueness of the density function of a distribution,
we will refer to ``the" density function of the distribution, with the
understanding that density functions could be changed slightly without
changing the distribution.

Let \(X\) denote a random variable with the uniform distribution on
\((0, 1)\). In Example \hyperref[uni_ex_den]{{[}uni\_ex\_den{]}} it is
shown that \(X\) has an continuous distribution with density function
\[p(x) = \begin{cases}1 & \text{ if } 0 \leq x \leq 1 \\
0 & \text{ otherwise} 
\end{cases}.\]

Note that the density function of \(X\) may also be taken to be
\[p_1(x) = \begin{cases}1 & \text{ if } 0 < x  <  1 \\
0 & \text{ otherwise}
\end{cases}.\]

That is, \[F(x) = \int_0^x p_1(t) dt, \ \ -\infty < x < \infty\] as
well.

In giving expressions for density functions, it is often convenient to
give the value of the density function, \(p(x)\), only for those values
of \(x\) for which the value is nonzero. For instance, in the previous
example, the density function might be given as \(p(x) = 1\),
\(0 < x < 1\). This statement implies that for \(x\geq 1\) or
\(x\leq 0\), \(p(x) = 0\).

The following theorem shows how the density function and distribution
are related for continuous distributions; it is essentially the
fundamental theorem of calculus, which relates differentiation and
integration.

\phantomsection\label{fund_theorem}{}

Let \(F\) denote the distribution function of a distribution on \(\Re\).

Suppose that \(F\) is continuous and there exists a function \(p\) such
that \[F'(x) = p(x) \ \ \hbox{ for almost all} \ \ x.\] Then \(p\) is
the density function corresponding to \(F\).

Let \(X\) denote a random variable with distribution function
\[F(x) = x^2(3 - 2x), \ \ 0 < x < 1.\] Note that \(F\) is continuous and
\[F'(x) = \begin{cases}6 (x - x^2) & \text{ if } 0 < x < 1 \\
                                   0 & \text{ otherwise}
\end{cases}.\]

It follows that the distribution is continuous with density function
\[p(x) = 6 (x - x^2), \ \ 0 < x < 1.\]

Let \(X\) denote a random variable with distribution function
\[F(x) = 1 - \exp(-x), \ \ x>0.\] Note that \(F\) is a continuous
function but that \(F'(x)\) does not exist at \(x=0\):
\[\lim_{h\to 0^+} \frac{F(h) - F(0)}{h} = \lim_{h\to 0^+} \frac{1 - \exp(-h)}{h} = 1\]
while
\[\lim_{h\to 0^-} \frac{F(h) - F(0)}{h} = \lim_{h\to 0^-} \frac{0}{h} = 0.\]

For \(x \neq 0\), \[F'(x) = \begin{cases} 0 & \text{ if } x<0 \\
                                 \exp(-x) & \text{ if } x>0
\end{cases}.\] Hence, the density of the distribution can be taken to be
\(\exp(-x)\), \(x>0\).

\subsection*{Interpretation of a density
function}\label{interpretation-of-a-density-function}
\addcontentsline{toc}{subsection}{Interpretation of a density function}

By the properties of integrals, if \(X\) has a continuous distribution
with density \(p\), then, for small \(\epsilon>0\),
\[\Pr(x_0 - \epsilon/2 < X < x_0 + \epsilon/2) = \int_{x_0-\epsilon/2}^{x_0 + \epsilon/2} p(x) 
dx \doteq p(x_0)\epsilon.\]

Hence, \(p(x)\) can be viewed as being proportional to the probability
that \(X\) lies in a small interval containing \(x\); it is important to
note that such an interpretation only gives an intuitive meaning to the
density function and cannot be used in formal arguments. It follows that
the density function gives an indication of the relative likelihood of
different possible values of \(X\).

When working with continuous distributions, density functions are
usually more informative than distribution functions for assessing the
basic properties of a probability distribution. Of course,
mathematically speaking, this statement is nonsense since the
distribution function completely characterizes a probability
distribution. However, for understanding the basic properties of the
distribution of random variable, the density function is often more
useful than the distribution function.

\phantomsection\label{df_den_ex}{}

Consider a continuous distribution with distribution function
\[F(x) = (5 - 2x)(x-1)^2, \  1 < x < 2\] and density function
\[p(x) = 6 (2-x)(x-1), \ \  1 < x < 2.\]

Figure \hyperref[df_den_plot]{1.3} gives a plot of \(F\) and \(p\).
Based on the plot of \(p\) it is clear that the most likely value of
\(X\) is \(3/2\) and, for \(z < 1/2\), \(X = 3/2 - z\) and
\(X =3/2 + z\) are equally likely; these facts are difficult to discern
from the plot of, or the expression for, the distribution function.

\begin{figure}

{\centering \includegraphics[width=0.75\linewidth,height=\textheight,keepaspectratio]{fig3.383.pdf}

}

\caption{Density and Distribution Functions in Example
\hyperref[df_den_ex]{{[}df\_den\_ex{]}}}

\end{figure}%

\subsection*{A paradox with continuous random
variables?}\label{a-paradox-with-continuous-random-variables}
\addcontentsline{toc}{subsection}{A paradox with continuous random
variables?}

Let \(Y\) denote random variable with a continuous distribution with
distribution function \(F\) and density function \(p\) and for a given
real number \(y\) consider \(\Pr(Y = y)\). Note that for any
\(\epsilon>0\),
\[\Pr(Y = y) \leq \Pr\left( Y\in (y-\epsilon, y]\right) = \Pr(y -\epsilon < Y \leq y) = F(y) - F(y-\epsilon)\]
and, hence,
\[\Pr(Y = y) \leq \lim_{\epsilon \to 0^+} \left(F(y) - F(y - \epsilon) \right) .\]
Because the distribution function of a continuous random variable is a
continuous function, the right-hand side of this expression is \(0\).
That is, \[\Pr( Y = y) = 0.\]

Because \(y\) is arbitrary we have the interesting result that, for a
continuous random variable \(Y\),
\[\Pr(Y = y) = 0 \ \ \text{ for all } \ \ y\in\Re.\] That is, although
\(Y\) takes values in \(\Re\), for all values of \(y\) in \(\Re\),
\(\Pr(Y = y) = 0\).

One explanation for this apparent paradox is based on the limiting
relative frequency interpretation of probability, as discussed in
Section \hyperref[prob_fcns]{{[}prob\_fcns{]}}. According to that
interpretation, \(\Pr(Y = y) = 0\) does not mean that \(Y = y\) is
absolutely impossible. What it means is that, if \(N_n(y)\) is the
number of times that \(Y=y\) occurs in \(n\) repetitions of the
experiment, then
\[\frac{N_n(y)}{n} \to 0 \ \ \text{ as } \ \ n\to \infty.\] Thus,
\(Y = y\) \textbf{might} occur, but it does not occur very often.

A useful implication of this result is that, for a continuous random
variable \(Y\),
\[\Pr(Y \leq y) = \Pr(Y < y \cup Y=y) = \Pr(Y < y) + \Pr(Y = y) = \Pr(Y < y);\]
similarly, \[\Pr(Y \geq y) = \Pr(Y > y).\]

\section{Expectation}\label{expect}

Let \(X\) denote a random variable with distribution function \(F\). The
expected value of \(X\), denoted by \(\mbox{E}(X)\), is a summary of the
distribution of \(X\); it represents a type of ``average value" of
\(X\). The expected value \(\mbox{E}(X)\) is often described as
the''mean of \(X\)" or as the ``mean of the distribution."

If \(X\) has a discrete distribution, taking the values
\(x_1, x_2, \ldots\) with mass function \(p\), then
\[\mbox{E}(X) = \sum_{j} x_j\ p(x_j);\] this can also be written
\[\mbox{E}(X)  =\sum_{x\in\mathcal{X}} x\, p(x).\] If \(X\) has a
continuous distribution with density \(p\), then
\[\mbox{E}(X) = \int_{-\infty}^\infty x\ p(x) dx.\]

Let \(X\) denote a random variable with a discrete distribution with
range \(\{0, 1, 2 \}\) and mass function \(p\) given by
\[p(x) = \begin{cases} \frac{1}{2} & \text{ if } \ x=0 \\
\frac{1}{4} & \text{ if } \ x=1 \\
\frac{1}{4} & \text{ if } \ x=2 
\end{cases}.\]

Then
\[\mbox{E}(X) = 0\left( \frac{1}{2}\right) + 1 \left( \frac{1}{4} \right) + 2 \left( \frac{1}{4}\right) = \frac{3}{4}.\]

\phantomsection\label{binom_ex5}{} Let \(X\) denote a random variable
with a binomial distribution with parameters \(n\) and \(\theta\), as
described in Example 1.4. Then \(X\) is a discrete random variable with
mass function
\[p(x) = {n \choose x} \theta^x (1-\theta)^{n-x}, \ \ \ x=0, \ldots, n\]
so that
\[\mbox{E}(X) = \sum_{j=0}^n j {n \choose j} \theta^j (1-\theta)^{n-j}.\]

Note that
\[j {n \choose j} = \frac{n!}{(j-1)! (n-j)!} = n {n-1 \choose j-1}.\]
Therefore, \[\begin{aligned}
\sum_{j=0}^n j {n \choose j} \theta^j (1-\theta)^{n-j} &=
\sum_{j=1}^n j {n \choose j} \theta^j (1-\theta)^{n-j} \\
&= n\theta \sum_{j=1}^n  {n-1 \choose j-1} \theta^{j-1} (1-\theta)^{n-j}\\
&= n\theta \sum_{j-1=0}^{n-1}  {n-1 \choose j-1} \theta^{j-1} (1-\theta)^{n-1-(j-1)}\\
&= n\theta \sum_{k=0}^{n-1}{n-1 \choose k}\theta^k (1-\theta)^{n-1-k}.
\end{aligned}\]

Note that the terms in the sum
\[\sum_{k=0}^{n-1}{n-1 \choose k}\theta^k (1-\theta)^{n-1-k}\] are the
values of the mass function of the binomial distribution with parameters
\(n-1\) and \(\theta\), evaluated at \(0, 1, \ldots, n-1\). It follows
that \[\sum_{k=0}^{n-1}{n-1 \choose k}\theta^k (1-\theta)^{n-1-k} = 1\]
and, hence, \[\mbox{E}(X) = n\theta.\]

Let \(X\) denote a random variable with a continuous distribution with
density function \[p(x) = \frac{2}{x^3}, \ \  x\geq 1.\]

Then
\[\mbox{E}(X) = \int_1^\infty x \frac{2}{x^{3}} dx =  \int_1^\infty 
\frac{2}{x^{2}} dx = -\frac{2}{x}\Bigm|_{1}^\infty = 2.\]

Let \(X\) denote a random variable with a continuous distribution with
density function \[p(x) = 20 x^3 (1-x), \ \ 0 < x < 1.\]

Then
\[\mbox{E}(X) = \int_0^1 x\, 20 x^3 (1-x) dx = 20 \int_0^1 (x^4 - x^5) dx = 20 \frac{1}{30} = \frac{2}{3}.\]

\subsection*{Expected value of a non-negative random
variable}\label{expected-value-of-a-non-negative-random-variable}
\addcontentsline{toc}{subsection}{Expected value of a non-negative
random variable}

Let \(X\) denote a random variable satisfying \(\mbox{P}(X\geq 0) = 1\).
Then \(\mbox{E}(X) \geq 0\).

To see why this holds, consider the case in which \(X\) has a continuous
distribution with density function \(p\). Because
\(\mbox{P}(X\geq 0) = 1\), we may assume that \(p(x) = 0\) for all
\(x\leq 0\). It follows that \[\mbox{E}(X) = \int_0^\infty x\, p(x) dx\]
and, because \(p(x)\geq 0\) for all \(x\), the integrand
\(x\, p(x) \geq 0\) for all \(x>0\). Hence, \(\mbox{E}(X)\) cannot be
negative.

A similar argument holds for a discrete random variable.

\subsection*{Existence of an expected
value}\label{existence-of-an-expected-value}
\addcontentsline{toc}{subsection}{Existence of an expected value}

There are three possibilities for an expected value \(\mbox{E}(X)\):
\(\mbox{E}(X) < \infty\), \(\mbox{E}(X) = \infty\), or \(\mbox{E}(X)\)
might not exist. In general, \(\mbox{E}(X)\) fails to exist if the
corresponding sum or integral used in its definition fails to exists.

If \(X\) is nonnegative, then \(\mbox{E}(X)\) always exists, although we
may have \(\mbox{E}(X) = \infty\). If \(X\) is discrete, \(\mbox{E}(X)\)
exists and is finite provided that \[\sum_{j} |x_j| \ p(x_j) < \infty;\]
if \(X\) is continuous, \(\mbox{E}(X)\) exists and is finite provided
that \[\int_{-\infty}^\infty|x|\ p(x) dx < \infty.\]

Let \(X\) denote a random variable with a continuous distribution with
density function \[p(x) = \frac{1}{x^2}, \ \  x\geq 1.\]

Then \[\mbox{E}(X) = \int_1^\infty x \frac{1}{x^{2}} dx = \int_1^\infty 
\frac{1}{x} dx = \log(x) \Bigm|_{1}^\infty = \infty.\]

Let \(X\) denote a random variable with a continuous distribution with
density function
\[p(x) = \frac{1}{\pi (1 + x^2)}, \ \  -\infty < x < \infty;\] this is a
\emph{standard Cauchy distribution}. If \(\mbox{E}(X)\) exists, it must
be equal to \[\begin{aligned}
 \int_{-\infty}^\infty \frac{x}{\pi(1 + x^2)} dx &= 
\int_0^\infty \frac{x}{ 
\pi(1 + x^2)} dx + \int_{-\infty}^0 \frac{x}{\pi(1 + x^2)} dx \\
&= \int_0^\infty \frac{x}{ 
\pi(1 + x^2)} dx - \int_0^\infty \frac{x}{\pi(1 + x^2)} dx. 
\end{aligned}\] Because
\[\int_0^\infty {x \over \pi(1 + x^2)} dx = \infty,\]
\[\int_{-\infty}^\infty \frac{x}{\pi(1 + x^2)} dx = \infty - \infty;\]
it follows that \(\mbox{E}(X)\) does not exist.

\section{Expectation of a Function of a Random
Variable}\label{expectation-of-a-function-of-a-random-variable}

Let \(X\) denote a random variable and let \(g\) denote a real-valued
function defined on the range of \(X\). Define \(Y= g(X)\) and consider
\(\mbox{E}(Y)\).

\subsection*{Discrete distributions}\label{discrete-distributions-1}
\addcontentsline{toc}{subsection}{Discrete distributions}

Consider the case in which \(X\) has a discrete distribution. The
following simple example illustrates the basic argument used in this
case.

Let \(X\) denote a discrete random variable with range
\(\mathcal{X}= \{-1, 0, 1\}\) and mass function \(p_X\) with
\(p_X(-1) = 0.1\), \(p_X(0) = 0.6\), and \(p_X(1) = 0.3\). Let
\(Y = X^2\). Then \(Y\) is a discrete random variable with range
\(\{0, 1\}\) and mass function \(p_Y\) with
\[p_Y(0) =\Pr(Y = 0) = \Pr(X^2=0) = \Pr(X = 0) = 0.6\] and
\[p_Y(1) = \Pr(Y = 1) =  \Pr(X^2 = 1) = \Pr(X = 1 \cup X=-1) = \Pr(X=1) + \Pr(X=-1) = 0.4.\]
Then \[\mbox{E}(Y) = \sum_y y p_Y(y) = 0(0.6) + 1(0.4) = 0.4.\]

As shown above, \(p_Y(0) = 0.6\) arises from the fact that \(x^2=0\)
only if \(x=0\) and \(p_Y(1) = 0.4\) arises from the fact that there are
two elements of \(\mathcal{X}\) that yield the value \(1\) for \(Y\):
\(-1\) and \(1\), with probabilities \(0.1\) and \(0.3\), respectively.

Hence, we can write
\[\mbox{E}(Y) = (0)^2 p_X(0) + (-1)^2 p_X(-1) + (1)^2 p_X(1);\] that is,
\[\mbox{E}(Y) = \sum_x x^2 p_X(x) .\] Alternatively, we can write
\[\mbox{E}(X^2) = \sum_x x^2 p_X(x)\] to avoid the need to define \(Y\).

This result holds in general:
\[\mbox{E}\left( g(X) \right) = \sum_{x\in \mathcal{X}} g(x) p_X(x).\]
The proof follows the argument given in the example. Let \(Y = g(X)\),
let \(p_Y\) denote the mass function \(Y\) and let \(\mathcal{Y}\)
denote its range. Then
\[\mbox{E}(Y) = \sum_{y\in \mathcal{Y}} y \ p_Y(y).\] Note that
\[p_Y(y) = \Pr(Y = y) = \Pr( X\in \{x\in\mathcal{X}: g(x) = y \}) = \sum_{x\in\mathcal{X}: g(x) = y} p_X(x).\]
Hence,
\[\mbox{E}(Y) = \sum_{y\in\mathcal{Y}} \sum_{x\in\mathcal{X}: g(x)=y} y\, p_X(x) = 
 \sum_{y\in\mathcal{Y}} \sum_{x: g(x)=y}  g(x) p_X(x).\] Because every
\(x\) value in the range of \(X\) leads to some value \(y\) in the range
of \(Y\), the summation
\[\sum_{y\in\mathcal{Y}} \sum_{x\in\mathcal{X}: g(x) = y}\] includes
every value of \(x\); thus,
\[\mbox{E}(Y) = \sum_{x\in\mathcal{X}}  g(x) p_X(x).\]

\phantomsection\label{uni_int}{} Let \(X\) denote a discrete random
variable with range \(\mathcal{X}= \{1, 2, \ldots, m \}\) for some
positive integer \(m\) and mass function \(p\) satisfying
\[p(1) = p(2) = \cdots = p(m) = \frac{1}{m};\] because all elements of
\(\mathcal{X}\) have the same probability, this is often referred to as
the \emph{uniform distribution on \(\mathcal{X}\)}.

Then, using the well-known result for the sum of first \(m\) positive
integers,
\[\mbox{E}(X) = \sum_{j=1}^m j \frac{1}{m} = \frac{1}{m}\sum_{j=1}^m j = \frac{1}{m}\frac{m(m+1)}{2} =\frac{m+1}{2}.\]
Similarly,
\[\mbox{E}(X^2) = \frac{1}{m} \sum_{j=1}^m j^2 = \frac{1}{m} \frac{m(m+1)(2m+1)}{6}
= \frac{(m+1)(2m+1)}{6},\] using the formula for the sum of the first
\(m\) squared positive integers.

\subsection*{Continuous distributions}\label{continuous-distributions}
\addcontentsline{toc}{subsection}{Continuous distributions}

Now consider the case in which \(X\) has a continuous distribution with
density \(p_X\) and \(g\) is a function on the range of \(\mathcal{X}\).
Then, by analogy with the discrete case,
\[\mbox{E}\left(g(X)\right) = \int_{-\infty}^\infty g(x) p_X(x) dx.\] A
formal proof of this result can be based on a general form of the
change-of-variable formula for integration and is beyond the scope of
this course.

Note that it causes no problem if \(g(x)\) is undefined for \(x \in A\)
for some set \(A\) such that \(\Pr(X\in A)=0\); this set can simply be
omitted when computing the expected value.

Let \(X\) denote a random variable with a continuous distribution with
density function \[p_X(x) = 3 x^2, \ \ 0 < x < 1\] and consider
\(\mbox{E}\left(X^{-1} \right)\). Then
\[\mbox{E}\left(X^{-1} \right) = \int_0^1 \frac{1}{x} 3 x^2\, dx = 3 \int_0^1 x\, dx = \frac{3}{2}.\]

\phantomsection\label{stdexp}{}

Let \(X\) denote a random variable with a continuous distribution with
density function \[p_X(x) = \exp( -x ), \ \ \ 0 < x < \infty.\]

Consider the expected value of \(X^r\) where \(r\) is a positive
integer. Then \[\label{exp_mom}
 \mbox{E}(X^r) = \int_0^\infty x^r \exp(-x) dx;\] this is simply an
expression for the well-known gamma function evaluated at \(r+1\),
\(\Gamma(r+1)\).

The gamma function is given by
\[\Gamma(z) = \int_0^\infty t^{z-1} \exp(-t) dt, \ \  z>0.\] It has the
properties that \(\Gamma(1) = 1\) and \[\Gamma(z+1) = z\Gamma(z).\]

Hence, if \(r\) is a positive integer, as in the present example,
\(\Gamma(r+1) = r!\). It follows that \[\mbox{E}(X^r) = r! ;\] this
result can also from successive applications of integration-by-parts in
evaluating the integral in \hyperref[exp_mom]{{[}exp\_mom{]}}.

Expected values of the form \(\mbox{E}(X^r)\) for \(r=1, 2, \ldots\) are
called the \emph{moments} of the distribution or the moments of \(X\).
Moments will be discussed in detail in Chapter 4.

\subsection*{Expectations of linear functions and
sums}\label{expectations-of-linear-functions-and-sums}
\addcontentsline{toc}{subsection}{Expectations of linear functions and
sums}

Suppose that \(X\) is a random variable, which may be either discrete or
continuous.

If \(g\) is a linear function of the form \(g(x) = ax + b\), for some
constants \(a, b\), then \(\mbox{E}\left(g(X) \right)\) can be obtained
directly from \(\\E(X)\): \[\mbox{E}(aX + b) = a \mbox{E}(X) + b.\] This
result follows directly from the properties of the sum or integral
defining the expectation. In particular, the expectation of a constant
is simply the constant: \(\mbox{E}(b) = b\).

Let \(X\) denote the random variable with range \(\{1, 2, \ldots, m\}\)
and mass function
\[p_X(x) = \frac{1}{m} \ \ \text{ for }  \ \ x=1, 2, \ldots, m.\] Note
this random variable was considered in Example
\hyperref[uni_int]{{[}uni\_int{]}}, where it was shown that
\(\mbox{E}(X) = (m+1)/2\).

Let \(Y = (X - 1)/m\). Then the range of \(Y\) is
\(\{0, 1/m, \ldots, (m-1)/m\}\) and, for \(y\) in this set, the mass
function of \(Y\) is given by \[p_Y(y) = \Pr(Y = y) = 1/m.\] It follows
that
\[\mbox{E}(Y) = \frac{1}{m}\mbox{E}(X) - \frac{1}{m} = \frac{m+1}{2m} - \frac{1}{m} = \frac{1}{2} - \frac{1}{2m}.\]

This result could also be found directly by computing the sum
\[\sum_{y=0, 1/m, \ldots, (m-1)/m}{\hspace{-30pt} y\, p_Y(y)} = \sum_{j=0}^{m-1} \frac{j}{m}\frac{1}{m}.\]

Suppose that the function \(g\) can be written as a sum of functions
\(g_1, g_2, \ldots, g_m\); that is,
\[g(x) = g_1(x) + g_2(x) + \cdots + g_m(x).\] Then
\[\mbox{E}\left(g(X)\right) =  \mbox{E}\left( g_1(X) + \cdots + g_m(X)\right)  = \mbox{E}\left( g_1(X) \right) + \cdots + \mbox{E}\left( g_m(X) \right) .\]
As with the previous result, this result follows from the properties of
the sums and integrals.

Let \(X\) denote a random variable with a continuous distribution with
density function \[p(x)  = \exp(-x), \ \ x>0.\] Find
\(\mbox{E}\left( \cosh(X/2) \right)\), where \(\cosh(x)\) is the
hyperbolic cosine function, given by
\[\cosh(x)= \frac{ \exp(x) + \exp(-x)}{2}.\]

Using the properties of the expectations of linear functions and sums,
\[\begin{aligned}
 \mbox{E}\left( \cosh(X/2) \right) &= \mbox{E}\left( \frac{ \exp(X/2) + \exp(-X/2)}{2} \right) \\
&= \frac{1}{2}  \mbox{E}\left( \exp(X/2) + \exp(-X/2) \right) \\
&= \frac{1}{2} \left( \mbox{E}\left( \exp(X/2)\right) + \mbox{E}\left(\exp(-X/2) \right) \right) . 
\end{aligned}\]

Note that
\[\mbox{E}\left( \exp(X/2) \right) = \int_0^\infty \exp(x/2) \exp(-x) dx = \int_0^\infty \exp(-x/2) dx = 2\]
and
\[\mbox{E}\left( \exp(-X/2) \right) = \int_0^\infty \exp(-x/2) \exp(-x) dx = \int_0^\infty \exp(-3x/2) dx = \frac{2}{3}.\]
It follows that
\[\mbox{E}\left( \cosh(X/2) \right) = \frac{1}{2} \left( 2 + \frac{2}{3}\right) = \frac{4}{3}.\]

\subsection*{Relationship between expectation and
probability}\label{relationship-between-expectation-and-probability}
\addcontentsline{toc}{subsection}{Relationship between expectation and
probability}

As might be expected, there is a close relationship between the
expectation of functions of a random variable and its probability
function.

For a set \(A\subset \Re\), define the \emph{indicator function} of
\(A\) by \[I_A(x) = \begin{cases} 1 & \text{ if } x\in A \\
                                      0 & \text{ if } x\notin A
\end{cases}.\]

Let \(X\) denote a random variable; then \(I_A(X)\) is a random variable
that takes two values \(1\), with probability \(\Pr(X \in A)\) and
\(0\), with probability \(\Pr(X \notin A)\). It follows that
\[\mbox{E}\left( I_A(X) \right) = 1\cdot \Pr(X\in A) + 0 \cdot \Pr(X \notin A) = \Pr( X\in A).\]
That is, probability is just a special case of expectation.

Suppose that \(X\) has a continuous distribution with density \(p\).
Then, using an integral to compute an expected value,
\[\Pr(X \in A) = \mbox{E}(I_A(X)) = \int_{-\infty}^\infty I_A(x) p(x)\, dx.\]
Because \(I_A(x)\) is \(1\) if \(x\in A\) and \(0\) otherwise,
\[\int_{-\infty}^\infty I_A(x) p(x)\, dx = \int_A p(x)\ dx,\] where
\[\int_A\] denotes integration over the set \(A\). Thus,
\[\Pr(X \in A) = \int_A p(x)\ dx,\] generalizing the expression for the
distribution function,
\[\Pr(X \leq x) \equiv \Pr(X \in (-\infty, x]) = \int_{-\infty}^x p(x)\, dx.\]

\subsection*{A useful inequality}\label{a-useful-inequality}
\addcontentsline{toc}{subsection}{A useful inequality}

Suppose that \(X\) is a nonnegative random variable:
\(\Pr(X \geq 0) = 1\). Then \(\mbox{E}(X) \geq 0\) and
\(\mbox{E}(X) = 0\) if and only if \(\Pr(X  = 0) = 1\).

This property suggests that if, for a nonnegative random variable \(X\),
\(\mbox{E}(X)\) is close to \(0\), then \(X\) must be close to \(0\)
with high probability. The following result, known as \emph{Markov's
inequality} gives a bound on \(\Pr(X \geq c)\) in terms of
\(\mbox{E}(X)\).

\phantomsection\label{markov}{} Let \(X\) denote a nonnegative random
variable. For any \(c > 0\),
\[\Pr( X \geq c ) \leq \frac{\mbox{E}(X)}{c}.\]

\begin{proof}
\emph{Proof.} Fix \(c>0\). Let \(A = [c, \infty)\) and and let
\(I_A(\cdot)\) denote the indicator function of the set \(A\).

Consider the function \[z - c I_A(z).\] Note that, if \(z \geq c\), then
\(I_A(z) = 1\) so that \[z - c I_A(z) = z-c \geq 0.\] If
\(0 \leq z < c\), then \(I_A(z) = 0\) so that
\[z - c I_A(z) = z \geq 0.\] It follows that, for \(z\geq 0\),
\[z - c I_A(z) \geq 0.\]

Let \(X\) denote a nonnegative random variable. Then \[X - c I_A(X)\] is
also a non-negative random variable and, hence,
\[\mbox{E}(X - cI_A(X)) = \mbox{E}(X) - c \mbox{E}(I_A(X)) \geq 0.\]
Using properties of indicator functions, it follows that
\[\mbox{E}(X) \geq c \mbox{E}\left( I_A(X) \right) = c \Pr( X \geq c)\]
or \[\Pr(X \geq c) \leq \frac{\mbox{E}(X)}{c}.\]~◻
\end{proof}

\phantomsection\label{Markov_ex}{} Let \(X\) denote a continuous random
variable with density \[p_X(x) =  2/x^3, \ \ x>1;\] we have seen that
\(\mbox{E}(X) = 2\).

According to Markov's inequality,
\[\Pr(X \geq c)  \leq \frac{2}{c} \ \ \text{ for }\ \ c>0.\]

For this distribution, we can calculate \(\Pr(X \geq c)\) exactly: for
\(c>1\),
\[\Pr(X \geq c) = \int_c^\infty \frac{2}{x^3} dx = \frac{2}{c^2}.\]

For instance, for \(c=10\), the bound based on Markov's inequality is
\(0.2\), while the true probability is \(0.02\).

Markov's inequality is most useful when \(\mbox{E}(X)\) is easy to
determine but \(\Pr(X \geq c)\) is not. If \(X\) is not necessarily
nonnegative, then Markov's inequality can be applied to \(|X|\) so that
\[\Pr(|X| \geq c) \leq \frac{\mbox{E}(|X|)}{c}.\]

If \(g\) is a nonnegative, strictly increasing, function, then
\[\Pr(X \geq c) = \Pr(g(X) \geq g(c)) \leq \frac{\mbox{E}\left( g(X) \right)}{g(c)}.\]

Let \(X\) denote a continuous random variable with density
\[p_X(x) =  2/x^3, \ \ x>1,\] as in Example
\hyperref[Markov_ex]{{[}Markov\_ex{]}}.

Note that, for \(r<2\), \[\begin{aligned}
 \mbox{E}(X^r) &= \int_1^\infty x^r \frac{2}{x^3} dx\\  &= 2 \int_1^\infty \frac{1}{x^{3-r}} dx \\
&= \frac{1}{2-r}. 
\end{aligned}\]

Then, for \(c>0\) and \(0 < r<2\),
\[\Pr(X \geq c) = \Pr( X^r \geq c^r) \leq \frac{ \mbox{E}\left( X^r \right)}{c^r} = \frac{1}{(2-r)c^r}.\]

For instance, for \(r=3/2\) and \(c=10\), this bound is about \(0.063\),
while the true probability is \(0.02\).

\section{Some Commonly-Used Families of
Distributions}\label{some-commonly-used-families-of-distributions}

Although a mass or density function can take virtually any form --
subject to the requirements that it is nonnegative and it either sums or
integrates to \(1\) -- there are certain families of distributions that
are often used in applications.

We have already seen one of these, the binomial distribution, studied in
Examples \hyperref[binom_ex2]{{[}binom\_ex2{]}},
\hyperref[binom_ex3]{{[}binom\_ex3{]}},
\hyperref[binom_ex4]{{[}binom\_ex4{]}}, and
\hyperref[binom_ex5]{{[}binom\_ex5{]}}. Recall the binomial distribution
is a discrete distribution with range
\(\mathcal{X}= \{0, 1, \ldots, n \}\) and mass function
\[p(x) = {n \choose x} \theta^x (1-\theta)^{n-x}, \ \ x=0, 1, \ldots, n.\]
Here \(n\) and \(\theta\) are parameters; \(n\) is a positive integer
and \(\theta\) is a real number in the interval \((0, 1)\).

Thus, every choice of parameter values represents a different
distribution; however, because the distributions for different parameter
values tend to have similar properties, it is convenient to consider
them together. The set of possible parameter values is known as the
\emph{parameter space} for the family.

In this section, we consider a few commonly-used families of
distribution; others will be introduced as needed throughout the course.

\subsection*{Gamma distributions}\label{gamma-distributions}
\addcontentsline{toc}{subsection}{Gamma distributions}

A random variable \(X\) is said to have a \emph{gamma distribution} with
parameters \(\alpha\) and \(\beta\) if it has a continuous distribution
with density function of the form \[\label{gamma_orig}
 p(x) = \frac{1}{\Gamma(\alpha) \beta^\alpha} x^{\alpha - 1} \exp(- x/\beta), \ \ x>0\]
where \(\alpha>0\) and \(\beta>0\).

Here \(\Gamma(\cdot)\) is the gamma function, discussed in Example
\hyperref[stdexp]{{[}stdexp{]}}. Recall that \(\Gamma(z)\) is defined
for \(z>0\) and the gamma function has the properties that
\(\Gamma(1) = 1\) and \[\Gamma(z+1) = z\Gamma(z).\] Hence, if \(r\) is a
positive integer, \(\Gamma(r+1) = r!\).

The gamma distribution illustrates a very useful property of parametric
families of density functions. Because we know that a density function
must integrate to \(1\), the form of the density function gives us an
integral identity: \[\label{gamma_ident}
 \int_0^\infty x^{\alpha - 1} \exp(- x/\beta) dx = \Gamma(\alpha) \beta^\alpha \ \text{ for all }\ \  \alpha>0, \ \ 
\beta>0.\]

It often happens that the integral identity based on a given density
function is useful for performing calculations related to that density.
For instance, suppose that \(X\) has a gamma distribution with
parameters \(\alpha\) and \(\beta\). Then
\[\mbox{E}(X) = \int_0^\infty x \frac{1}{\Gamma(\alpha) \beta^\alpha} x^{\alpha - 1} \exp(- x/\beta) dx.\]
Note that the integrand in this expression is of a form closely related
to that of a gamma density:
\[\int_0^\infty x \frac{1}{\Gamma(\alpha) \beta^\alpha} x^{\alpha - 1} \exp(- x/\beta) dx = 
 \int_0^\infty \frac{1}{\Gamma(\alpha) \beta^\alpha} x^{\alpha} \exp(- x/\beta) dx.\]
Using the identity \hyperref[gamma_ident]{{[}gamma\_ident{]}},
\[\begin{aligned}
 \int_0^\infty \frac{1}{\Gamma(\alpha) \beta^\alpha} x^{\alpha} \exp(- x/\beta) dx
&=  \frac{1}{\Gamma(\alpha) \beta^\alpha}  \int_0^\infty  x^{\alpha} \exp(- x/\beta) dx\\
&= \frac{1}{\Gamma(\alpha) \beta^\alpha}  \Gamma(\alpha+1) \beta^{\alpha+1} \\
&= \frac{\Gamma(\alpha+1)}{\Gamma(\alpha)} \beta.
\end{aligned}\] Using the reproductive property of the gamma function,
\(\Gamma(\alpha+1) = \alpha\Gamma(\alpha)\), so that
\[\mbox{E}(X) = \alpha\beta.\]

Often, the density functions of the family of gamma distributions are
taken to be of the form \[\label{gamma_alt}
p(x) = \frac{\beta^{\alpha}}{\Gamma(\alpha)} x^{\alpha - 1} \exp(- \beta x), \ \ x>0\]
where \(\alpha>0\) and \(\beta>0\). Note that the density functions
given in equation \hyperref[gamma_orig]{{[}gamma\_orig{]}} and those
given in \hyperref[gamma_alt]{{[}gamma\_alt{]}} describe the same family
of distributions.

For example, the density function
\[p(x) = \frac{1}{\Gamma(3) 2^3} x^{2} \exp(- x/2)\] is of the form
given by equation \hyperref[gamma_orig]{{[}gamma\_orig{]}} with
\(\alpha = 3\) and \(\beta = 2\) and it is also of the form given by
equation \hyperref[gamma_alt]{{[}gamma\_alt{]}} with \(\alpha = 3\) and
\(\beta = 1/2\).

In the alternative parameterization given by
\hyperref[gamma_alt]{{[}gamma\_alt{]}},
\[\mbox{E}(X) = \frac{\alpha}{\beta}.\]

Because both forms of the density function are routinely used, when
using results regarding the gamma distribution, it is important to keep
track of which form of the density function is under consideration.

\subsection*{Exponential distribution}\label{exponential-distribution}
\addcontentsline{toc}{subsection}{Exponential distribution}

A special case of the gamma distribution is the \emph{exponential
distribution}, which is a gamma distribution with parameter \(\alpha\)
taken to be \(1\). Often (but not always), the parameter \(\beta\) of
the gamma distribution is written in terms of \(\lambda =1/\beta\), so
that a random variable \(X\) is said to have an exponential distribution
with parameter \(\lambda\) if it has a continuous distribution with
density function of the form
\[p(x) = \lambda \exp(-\lambda x), \ \ x>0\] where \(\lambda>0\).

A \emph{standard} exponential distribution has \(\lambda = 1\); thus,
the density of the standard exponential distribution is
\(\exp(-x), x>0\).

\subsection*{Gaussian distribution}\label{gaussian-distribution}
\addcontentsline{toc}{subsection}{Gaussian distribution}

A random variable \(X\) is said to have a \emph{Gaussian distribution}
with parameters \(\mu\) and \(\sigma^2\) if it has a continuous
distribution with density function of the form
\[p(x) = \frac{1}{\sqrt{2\pi\sigma^2}} \exp\left( - \frac{1}{2\sigma^2}(x - \mu)^2 \right), \ \ 
-\infty < x < \infty\] where \(-\infty < \mu < \infty\) and
\(\sigma^2>0\). Note that \(\sigma = \sqrt{\sigma^2}\) is sometimes used
in place of \(\sigma^2\) as a parameter; then \(\sigma>0\). Note that
the family of distributions does not change when this switch is made.
The Gaussian distribution (also called the \emph{normal distribution})
is the most commonly-used distribution in statistics.

The \emph{standard Gaussian distribution} is the Gaussian distribution
with parameter values \(\mu = 0\) and \(\sigma^2 = 1\); that is, the
standard Gaussian distribution has density function
\[\frac{1}{\sqrt{2\pi}} \exp\left( - \frac{1}{2}x^2 \right), \ \ 
-\infty < x < \infty.\]

The density function of the Gaussian distribution is often described as
being a ``bell-shaped curve"; see Figure \hyperref[gauss_fig1]{1.4} for
a plot of the standard Gaussian density.

\begin{figure}

{\centering \includegraphics[width=0.75\linewidth,height=\textheight,keepaspectratio]{stdgauss.pdf}

}

\caption{Density Function of the Standard Gaussian Distribution}

\end{figure}%

An important property of the standard Gaussian density is that it is
symmetric about \(0\); this can be seen from the form of the density
given previously, as well as from the plot of the density in Figure
\hyperref[gauss_fig1]{1.4}. One consequence of this property is that, if
\(X\) has a standard Gaussian distribution, then
\[\Pr(X > c) = \Pr(X < -c)\] for any real number \(c\). This is
illustrated in Figure \hyperref[gauss_fig2]{1.5} for the case \(c=1\).

For the general Gaussian distribution, with parameter \(\mu\) and
\(\sigma\), the density function is symmetric about \(x=\mu\), so that
\[\Pr(X > \mu + c) = \Pr(X < \mu - c)\] for any real number \(c\).

\subsection*{Poisson distribution}\label{poisson-distribution}
\addcontentsline{toc}{subsection}{Poisson distribution}

A random variable \(X\) is said to have a \emph{Poisson distribution}
with parameter \(\lambda\) if it has a discrete distribution with range
\(\mathcal{X}= \{0, 1, 2, \ldots\}\) and mass function
\[p(x) = \frac{\lambda^x \exp(-\lambda)}{x!}, \ \ x=0, 1, 2, \ldots\]
where \(\lambda>0\).

Just as the density function of a continuous distribution gives an
integral identity, the mass function of discrete distribution gives an
identity for a certain type of sum. For the Poisson distribution, this
identity states that
\[\sum_{x=0}^\infty \frac{\lambda^x \exp(-\lambda)}{x!} = 1\] or,
equivalently,
\[\sum_{x=0}^\infty \frac{\lambda^x }{x!} = \exp(\lambda), \ \ \text{ for all }\ \ \lambda>0.\]

\subsection*{Distribution table}\label{distribution-table}
\addcontentsline{toc}{subsection}{Distribution table}

The final page of this chapter contains a table of some commonly-used
distributions. Note that the variance and the ``mgf'' (i.e., the
moment-generating function) of a distribution will be discussed later in
the course.

\begin{figure}

{\centering \includegraphics[width=0.75\linewidth,height=\textheight,keepaspectratio]{gausscomp.pdf}

}

\caption{Comparison of Probabilities for the Standard Gaussian
Distribution}

\end{figure}%

\bookmarksetup{startatroot}

\chapter*{References}\label{references}
\addcontentsline{toc}{chapter}{References}

\markboth{References}{References}

\phantomsection\label{refs}




\end{document}
